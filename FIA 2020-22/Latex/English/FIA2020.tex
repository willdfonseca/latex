%%%%%%%%%%%%%%%%%%%%%%%%%%%%%%%%%%%%%%%%%%%%%%%%%%%%%%%%%%%%%%%%%%%%%%%%%%%%%%%%%%%%%%%%%%%%%%%%%%%%%%%%%%%%%%%%%%%%%%%%%%%%%%
%%% LaTeX template for the 12º Congresso Iberoamericano de Acústica (12º Congreso Iberoamericano de Acústica) and XXIX Encontro da Sobrac (XXIX Sobrac Meeting)
%%% Based on the Acústica e Vibrações (Sobrac) template
%%% Release 19/04/2022
%%%	Develped by Prof. William D'Andrea Fonseca, Dr. Eng. - Engenharia Acústica UFSM, English revision and translation by Thiago Morphy and Stephan Paul
%%% will.fonseca@eac.ufsm.br
%%%%%%%%%%%%%%%%%%%%%%%%%%%%%%%%%%%%%%%%%%%%%%%%%%%%%%%%%%%%%%%%%%%%%%%%%%%%%%%%%%%%%%%%%%%%%%%%%%%%%%%%%%%%%%%%%%%%%%%%%%%%%%
\documentclass[12pt, a4paper, twoside, twocolumn]{article}
%%%%%%%%%%%%%%%%%%%%%%%%%%%%%%%%%%%%%%%%%%%%%%%%%%%%%%%%%%
%%% Input
\usepackage[utf8]{inputenc} 
\usepackage[english]{babel} % Select the options that fit your needs.
\usepackage[T1]{fontenc}
\usepackage{lmodern, mathptmx} % For Times New Roman
%%%%%%%%%%%%%%%%%%%%%%%%%%%%%%%%%%%%%%%%%%%%%%%%%%%%%%%%%%%%%%%%%%%%%%%%%%%%%%%%%%%%%%%%%%%%%%%%%%%%%%%%%%%%%%%%%%%%
\usepackage{FIA2020} %%% Template basics
%%%%%%%%%%%%%%%%%%%%%%%%%%%%%%%%%%%%%%%%%%%%%%%%%%%%%%%%%%%%%%%%%%%%%%%%%%%%%%%%%%%%%%%%%%%%%%%%%%%%%%%%%%%%%%%%%%%%
%%% Select language options

% For spanish, uncomment the line below
%\Resumen

% For English articles, please do not comment the following line
\SuppressResumo 
%%%%%%%%%%%%%%%%%%%%%%%%%%%%%%%%%%%%%%%%%%%%%%%%%%%%%%%%%%%%%%%%%%%%%%%%%%%%%%%%%%%%%%%%%%%%%%%%%%%%%%%%%%%%%%%%%%%%
%%% Paper data

\Authors{Fonseca,~W.~D'A.; Last~name,~I.} % For PDF metadata

% Use the structure Last name, I
\AuthorsAffiliations{Fonseca,~W.~D'A.$^1$; Last~name,~N.$^2$} % Use numbers to mark the affiliations

%% Use one line for each author of different affiliations. 
%% If authors share the same affiliation, use the same data and declare different emails.
\Affiliations{$^1$\,Acoustic Engineering, Universidade Federal de Santa Maria, Santa Maria, RS, Brasil, will.fonseca@eac.ufsm.br\\[2pt]  
$^2$\,Vibration Laboratory, Institution, City, State, Country, name@domain.com}

\TitleComplete{Instructions and article template for the FIA 2020/22 and\\ XXIX Sobrac meeting}
\TitleShort{Instructions and article template FIA 2020/22 and XXIX Sobrac}              % Short title for the heaing
\TitleEnglish{Instructions and article template for the FIA 2020/22 and XXIX Sobrac meeting} % Additional Title in English for papers written in Portuguese or Spanish

%\PalavrasChave{artigo técnico, FIA, Sobrac, acústica, vibrações} % o Palavras claves
\Keywords{technical paper, FIA, Sobrac, acoustics, vibration}

\PACS{please see the instructions in this template}


\Abstract{
This field is intended for the article's  abstract, that should have 180 to 300 words. Title, abstract, keywords and PACS should feature be on the  first page (i.e. avoid extending them to the following page). The abstract should make a concise presentation of the scientific-technical article, containing an introduction, the objective, a synthesis of the method, the main result and the final conclusion (preferably in that order). No separate items or sections are required within the abstract. Thus, the reader may acknowledge the essence of the article's content. Remember that the abstract is like a movie trailer, people will consider reading the complete article if the abstract is interesting. The abstract should not contain new information not contained within the article; undefined abbreviations; previous discussion of the literature; references and citations or excessive detail about the methods employed. It is also not the introductory paragraph of the work; this should be placed at the beginning of the text. Use only relevant and useful information, exercising empathy with prospective readers. For a cohesive, elegant abstract that represents the article, write a preview, write the paper completely, and then review it by looking at whether its content consistently reflects the content of the document. Following the abstract, the author should list up to five keywords (avoid using the same words contained in the article’s title). PACS identifiers, a hierarchical classification system (more details within the text) should be given too.}

\Metadata % Includes the metada into the PDF
%%%%%%%%%%%%%%%%%%%%%%%%%%%%%%%%%%%%%%%%%%%%%%%%%%%%%%%%%%%%%%%%%%%%%%%%%%%%%%%%%%%%%%%%%%%%%%%%%%%%%%%%%%%%%%%%%%%%
%%%%%%%%%%%%%%%%%%%%%%%%%%%%%%%%%%%%%%%%%%%%%%%%%%%%%%%%%%%%%%%%%%%%%%%%%%%%%%%%%%%%%%%%%%%%%%%%%%%%%%%%%%%%%%%%%%%%
\begin{document} \setcounter{page}{1} %%%%%%%%%%%%%%%%%%%%%%%%%%%%%%%%%%%%%%%%%%%%%%%%%%%%%%%%%%%%%%%%%%%%%%%%%%%%%%%%%%%%%%%%%%%%%%%%%%%%%%%%%%%%%%%%%%%%
%%% Template de LaTeX para o Congresso Acústica 2020, que integra o XI Congreso Ibérico da Acústica, 
%%% 51º Congresso Espanhol de Acústica e TecniAcústica 2020
%%% Release 06/08/2020
%%%	Desenvolvido por Prof. William D'Andrea Fonseca da Engenharia Acústica, UFSM, Brasil
%%% will.fonseca@eac.ufsm.br
%%%%%%%%%%%%%%%%%%%%%%%%%%%%%%%%%%%%%%%%%%%%%%%%%%%%%%%%%%%%%%%%%%%%%%%%%%%%%%%%%%%%%%%%%%%%%%%%%%%%%%%%%%%%%%%%%%%%
%% Estilo do artigo
\pagestyle{plain}
%%%%%%%%%%%%%%%%%%%%%%%%%%%%%%%%%%%%%%%%%%%%%%%%%%%%%%%%%%%%%%%%%%%%%%%%%%%%%%%%%%%%%%%%%%%%%%%%%%%%%%%%%%%%%%%%%%%%
%%% Primeira página
\thispagestyle{firststyle}
%%%%%%%%%%%%%%%%%%%%%%%%%%%%%%%%%%%%%%%%%%%%%%%%%%%%%%%%%%%%%%%%%%%%%%%%%%%%%%%%%%%%%%%%%%%%%%%%%%%%%%%%%%%%%%%%%%%%
%%% Título
\begin{textblock}{210}(0,10)
\begin{figure}
\centering \includegraphics[height=2cm,page=1]{ac2020.pdf}
\end{figure}
\end{textblock}

\vspace{1cm}
\begin{center}
{\fontsize{14}{16}\selectfont\bfseries \MakeUppercase
%% Título
%%%%%%%%%%%%%%%%%%%%%%%%%%%%%%%%%%%%%%%%%%%%%%%%%%%%%%%%%%%%%%%%%%%%%%%%%%%%%%%%%%%%%%%%%%%%%%%%%%%%%%%%%%%%%%%%%%%%
\TituloCompletoArtigo
%%%%%%%%%%%%%%%%%%%%%%%%%%%%%%%%%%%%%%%%%%%%%%%%%%%%%%%%%%%%%%%%%%%%%%%%%%%%%%%%%%%%%%%%%%%%%%%%%%%%%%%%%%%%%%%%%%%%
\par}

%%%%%%%%%%%%%%%%%%%%%%%%%%%%%%%%%%%%%%%%%%%%%%%%%%%%%%%%%%%%%%%%%%%%%%%%%%%%%%%%%%%%%%%%%%%%%%%%%%%%%%%%%%%%%%%%%%%
%%%%%%%%%%%%%%%%%%%%%%%%%%%%%%%%%%%%%%%%%%%%%%%%%%%%%%%%%%%%%%%%%%%%%%%%%%%%%%%%%%%%%%%%%%%%%%%%%%%%%%%%%%%%%%%%%%%
\vspace{12pt}
{\fontsize{10}{12}\selectfont \bfseries 
%% Autores
%%%%%%%%%%%%%%%%%%%%%%%%%%%%%%%%%%%%%%%%%%%%%%%%%%%%%%%%%%%%%%%%%%%%%%%%%%%%%%%%%%%%%%%%%%%%%%%%%%%%%%%%%%%%%%%%%%%
\AutoresFiliacoesArtigo
%%%%%%%%%%%%%%%%%%%%%%%%%%%%%%%%%%%%%%%%%%%%%%%%%%%%%%%%%%%%%%%%%%%%%%%%%%%%%%%%%%%%%%%%%%%%%%%%%%%%%%%%%%%%%%%%%%%
\par}

\vspace{2mm}
{\fontsize{10}{11}\selectfont 
%% Filiações
%%%%%%%%%%%%%%%%%%%%%%%%%%%%%%%%%%%%%%%%%%%%%%%%%%%%%%%%%%%%%%%%%%%%%%%%%%%%%%%%%%%%%%%%%%%%%%%%%%%%%%%%%%%%%%%%%%%
\FiliacoesArtigo
%%%%%%%%%%%%%%%%%%%%%%%%%%%%%%%%%%%%%%%%%%%%%%%%%%%%%%%%%%%%%%%%%%%%%%%%%%%%%%%%%%%%%%%%%%%%%%%%%%%%%%%%%%%%%%%%%%%%
\par}
\end{center}
%%%%%%%%%%%%%%%%%%%%%%%%%%%%%%%%%%%%%%%%%%%%%%%%%%%%%%%%%%%%%%%%%%%%%%%%%%%%%%%%%%%%%%%%%%%%%%%%%%%%%%%%%%%%%%%%%%%%
%%%%%%%%%%%%%%%%%%%%%%%%%%%%%%%%%%%%%%%%%%%%%%%%%%%%%%%%%%%%%%%%%%%%%%%%%%%%%%%%%%%%%%%%%%%%%%%%%%%%%%%%%%%%%%%%%%%%
%%%%%%%%%%%%%%%%%%%%%%%%%%%%%%%%%%%%%%%%%%%%%%%%%%%%%%%%%%%%%
%%% Resumo e palavras-chave
%%%%%%%%%%%%%%%%%%%%%%%%%%%%%%%%%%%%%%%%%%%%%%%%%%%%%%%%%%%%%
		{\fontsize{11}{13}\selectfont 
		\textbf{Resumo}
		\vspace{-2mm}
		
		\ResumoArtigo
		\par}
		%%%%%%%%%%%%%%%%%%%%%%%%% Palavras-chave
		\vspace{0.1\baselineskip} \fontsize{11}{13}\selectfont 
		\textbf{Palavras-chave: }{\fontsize{11}{13}\selectfont 
		\PalavrasChaveArtigo.
		\par}
%%%%%%%%%%%%%%%%%%%%%%%%%%%%%%%%%%%%%%%%%%%%%%%%%%%%%%%%%%%%%%%%%%%%%%%%%%%%%%%%%%%%%%%%%%%%%%%%%%%%%%%%%%%%%%%%%%%%
%%%%%%%%%%%%%%%%%%%%%%%%%%%%%%%%%%%%%%%%%%%%%%%%%%%%%%%%%%%%%%%%%%%%%%%%%%%%%%%%%%%%%%%%%%%%%%%%%%%%%%%%%%%%%%%%%%%%
\vspace{0.5\baselineskip}
%%%%%%%%%%%%%%%%%%%%%%%%%%%%%%%%%%%%%%%%%%%%%%%%%%%%%%%%%%%%%
%%% Abstract and keywords
%%%%%%%%%%%%%%%%%%%%%%%%%%%%%%%%%%%%%%%%%%%%%%%%%%%%%%%%%%%%%
\begin{otherlanguage*}{english}
%%%%%%%%%%%%%%%%%%%%%%%%%%%%%%%%%%%%%%%%%%%%%%%%%%%%%%%%%%%%%%%%%%%%%%%%%%%%%%%%%%%%%%%%%%%%%%%%%%%%%%%%%%%%%%%%%%%%
%%%%%%%%%%%%%%%%%%%%%%%%%%%%%%%%%%%%%%%%%%%%%%%%%%%%%%%%%%%%%%%%%%%%%%%%%%%%%%%%%%%%%%%%%%%%%%%%%%%%%%%%%%%%%%%%%%%%
%%% Abstract
		{\fontsize{11}{13}\selectfont 
		\textbf{Abstract}
		\vspace{-2mm}
		
		\AbstractArtigo
		\par}
		%%%%%%%%%%%%%%%%%%%%%%%%% Keywords
		\vspace{0.1\baselineskip} \fontsize{11}{13}\selectfont 
		\textbf{Keywords: }{\fontsize{11}{13}\selectfont 
		\KeywordsArtigo.
		\par}
\end{otherlanguage*}		
%%%%%%%%%%%%%%%%%%%%%%%%%%%%%%%%%%%%%%%%%%%%%%%%%%%%%%%%%%%%%%%%%%%%%%%%%%%%%%%%%%%%%%%%%%%%%%%%%%%%%%%%%%%%%%%%%%%%
%%%%%%%%%%%%%%%%%%%%%%%%%%%%%%%%%%%%%%%%%%%%%%%%%%%%%%%%%%%%%%%%%%%%%%%%%%%%%%%%%%%%%%%%%%%%%%%%%%%%%%%%%%%%%%%%%%%%
%%%%%%%%%%%%%%%%%%%%%%%%% PACs:
		\vspace{0.1\baselineskip}
		{\fontsize{11}{13}\selectfont \bfseries
		\textbf{PACS no. }
		\PACSArtigo.
		\par}
%%%%%%%%%%%%%%%%%%%%%%%%%%%%%%%
\vspace{1\baselineskip}
% EOF %%%%%%%%%%%%%%%%%%%%%%%%%



%%%%%%%%%%%%%%%%%%%%%%%%%%%%%%%%%%%%%%%%%%%%%%%%%%%%%%%%%%%%%%%%%%%%%%%%%%%%%%%%%%%%%%%%%%%%%%%%%%%%%%%%%%%%%%%%%%%%
%%%%%%%%%%%%%%%%%%%%%%%%%%%%%%%%%%%%%%%%%%%%%%%%%%%%%%%%%%%%%%%%%%%%%%%%%%%%%%%%%%%%%%%%%%%%%%%%%%%%%%%%%%%%%%%%%%%%
%%% ARTICLE
%%%%%%%%%%%%%%%%%%%%%%%%%%%%%%%%%%%%%%%%%%%%%%%%%%%%%%%%%%%%%%%%%%%%%%%%%%%%%%%%%%%%%%%%%%%%%%%%%%%%%%%%%%%%%%%%%%%%
%%%%%%%%%%%%%%%%%%%%%%%%%%%%%%%%%%%%%%%%%%%%%%%%%%%%%%%%%%%%%%%%%%%%%%%%%%%%%%%%%%%%%%%%%%%%%%%%%%%%%%%%%%%%%%%%%%%%
\clearpage % It is recommended to leave only the summary data of the first page, however, this is not mandatory.

\section{Introduction}

This template instruction text was elaborated so that authors can elaborate their articles in a standardized way. The text was adapted from the ``Acústica e Vibrações'' (from Sobrac) journal template, to be used for the 12º Iberoamerican Acoustics Congress integrated with the XXIX Sobrac Meeting. Templates are thought to provide an uniform formatting for all articles of the event. Therefore, in this template, the main guidelines for article elaboration regarding content, graphics, structure, layout presentation and submission are presented. The template implements the custom styles to format the article properly. The author can, therefore, use this file as a template or model for his article. In addition to the present \LaTeX\xspace (\texttt{.tex}) template a  Microsoft Word (\texttt{.docx}) template will be available. This version is also available on \href{https://www.overleaf.com/read/hgryywpgmxdx}{Overleaf} and \href{https://github.com/willdfonseca/fia2020}{GitHub}, and is compatible with Windows, Mac and Linux. Depending on the set-up of your TeX distribution you might be required to download and install additional packages or fonts if you decide to compile locally on your machine. 
Authors are responsible for the article's content, elaboration and submission in agreement with the present template.

The complete text shall use simple line spacing, using 12-pt Times New Roman font and 0-pt spacing before and 12-pt after paragraphs. The template will take care of this automatically. 
It is common practice to write scientific articles in the impersonal, therefore its practice is recommended. Articles can be written in Portuguese, English\footnote{Articles written in English by non-native speakers should, by preference, pass a professional revision.} and Spanish.

%%%%%%%%%%%%%%%%%%%%%%%%%%%%%%%%%%%%%%%%%%%%%%%%%%%%%%%%%%%%%%%%%%%%%%%%%%%%%%%%%%%%%%%%%%%%%%%%%%%%%%%%%%%%%%%%%%%
%%%%%%%%%%%%%%%%%%%%%%%%%%%%%%%%%%%%%%%%%%%%%%%%%%%%%%%%%%%%%%%%%%%%%%%%%%%%%%%%%%%%%%%%%%%%%%%%%%%%%%%%%%%%%%%%%%%
\section{Basic orientations}

In this section, a summary of how the article must be structured is presented. For more details, check the specific sections in this template.

\vspace{-8pt}
\begin{enumerate} \itemsep=2pt
    \item The provided \LaTeX{} and Word templates contain all configurations required for proper formatting that are  described in this document. Moreover, this text provides simultaneously instructions for both writing software.
    \item The first page of an article written in English should feature the title, authors, affiliations, abstract, keywords, and PACS.
    \item The text must be written using  standard language's norms.
    \item The maximum number of pages is 12, including the title page and pages for appendices, if any.
    \item Paper size is A4, with the following margins: 2.1~cm from the top, 2.0~cm from the bottom, 1.9~cm from the left and also 1.9~cm from the right (spacing between columns is 1.1cm).
    \item Text must be written in 12-pt Times New Roman, as is in this template.
    \item The article can contain figures, tables, boards, codes and equations to be placed in the running text. In the text, if necessary, links are allowed to be inserted. Animations are also allowed, as long as being represented by diagrams in figures. 
    \item A technical article is expected to have a logical, descriptive structure with reproducible content and a list of all references cited in the text.
\end{enumerate}

%%%%%%%%%%%%%%%%%%%%%%%%%%%%%%%%%%%%%%%%%%%%%%%%%%%%%%%%%%%%%%%%%%%%%%%%%%%%%%%%%%%%%%%%%%%%%%%%%%%%%%%%%%%%%%%%%%%
\section{Document and presentation}

Always insert text between sections or subsections, do not orphan them (beginning a section and going directly to the subsection)

%%%%%%%%%%%%%%%%%%%%%%%%%%%%%%%%%%%%%%%%%%%%%%%%%%%%%%%%%%%%%%%%%%%%%%%%%%%%%%%%%%%%%%%%%%%%%%%%%%%%%%%%%%%%%%%%%%%
\subsection{First page}

The first page shall contain the following items to be completed by the authors: title, authors' names, affiliations, abstract, PACS and keywords. If the complete title is too long, a shorter version is requested to be included in the header of the articles' pages.

The abstract should have between 180 and 300 words. Make sure that title, authors' names, affiliations, abstract, PACS and keywords fit on the first page. The abstract should make a concise presentation of the scientific-technical article, containing an introduction, the objective, a synthesis of the methodology, the main result and the final conclusion (preferably in that order). No separate items or sections are required within the abstract. The reader should be able to capture the essence of the article's content. Remember that the abstract is like a movie trailer, people will consider reading the complete article if the abstract is interesting. The abstract should not contain information not contained within the article. Avoid using undefined abbreviations; making discussions of the literature; including references and citations or excessive detail about the methods employed. It is also not the introductory paragraph of the work; the introduction is to be provided at the beginning of the main text on the next page. Use only relevant and useful information, exercising empathy with prospective readers. For a cohesive, elegant abstract that represents the article, write a preview, write the paper completely, and then review it by looking at whether its content consistently reflects the content of the document.

Following the abstract, the author should list up to five keywords. Avoid using the same words contained in the article’s title.

After that, there is still the 3--5 PACS (Physics and Astronomy Classification Scheme) code presentation, which is a hierarchical classification system created by the American Institute of Physics (AIP). It aids in identifying fields and sub-fields in physics and related subjects. This classification is used in international journal articles, as well as for some articles to be published in conference proceedings. PACS codes are composed by numbers and letters, e.g., ``43.20.Dk'' for ``Ray acoustic''. The authors should look for for the best classification provided  by the Journal of the Acoustical Society of America at:

\begin{itemize}[noitemsep,topsep=-1ex] \itemsep=8pt
	\item \url{https://asa.scitation.org/jas/authors/manuscript}
	\item \url{https://asa.scitation.org/pb-assets/files/publications/jas/Acoustics_PACS-1548697226033.pdf}
\end{itemize}

The PACS codes should be placed following the abstract.

For the authors' affiliations, use numbers as superscripts. If there are multiple authors with the same affiliation, use only one address but add the the different e-mails. When the email domain  addresses are the same too, try to shorten them using braces $\{ \}$. Use a maximum of two lines for each author affiliation. See some of the following examples:
%
\begin{flushleft}
\vspace{-0.5\baselineskip}
\begin{itemize}[topsep=-1ex,align=left,leftmargin=0.2cm] \itemsep=4pt

	\item Fonseca,~W.~D'A.$^1$; Last name,~N.$^2$\\[6pt]	
	$^{1,2}$\,Acoustic Engineering, Universidade Federal de Santa Maria, Santa Maria, RS, Brasil,\linebreak 
	 will.fonseca@eac.ufsm.br, name@domain.com.
	
	\item Fonseca,~W.~D'A.$^1$; Mareze,~P.~H.$^2$\\[6pt]	
	$^{1-2}$\,Acoustic Engineering, Universidade Federal de Santa Maria, Santa Maria, RS, Brasil,
	\{will.fonseca, paulo.mareze\}@eac.ufsm.br.
	
	\item Fonseca,~W.~D'A.$^1$; Last name,~N.$^2$, Mareze,~P.~H.$^3$\\[6pt]	
	$^{1,3,2}$\,Acoustic Engineering, Universidade Federal de Santa Maria, Santa Maria, RS, Brasil,
	\{will.fonseca, paulo.mareze\}@eac.ufsm.br,\linebreak name@domain.com.

	\item Fonseca,~W.~D'A.$^1$; Last name,~N.$^2$\\[6pt]	
	$^{1}$\,Acoustic Engineering, Universidade Federal de Santa Maria, Santa Maria, RS, Brasil,
	will.fonseca@eac.ufsm.br.\\[4pt]		
	$^2$\,Vibration laboratory, Institution, City, State, Country, name@domain.com.	
\end{itemize}
\vspace{-0.4\baselineskip}
\end{flushleft}

%%%%%%%%%%%%%%%%%%%%%%%%%%%%%%%%%%%%%%%%%%%%%%%%%%%%%%%%%%%%%%%%%%%%%%%%%%%%%%%%%%%%%%%%%%%%%%%%%%%%%%%%%%%%%%%%%%%
\subsection{Number of pages}

The complete work should not exceed 12 pages, including the title page, the complete list of references and appendices, if there are any.

To optimize the space available, figures, tables and codes must be presented within the body  of the text, using one or two columns depending on their content.

%%%%%%%%%%%%%%%%%%%%%%%%%%%%%%%%%%%%%%%%%%%%%%%%%%%%%%%%%%%%%%%%%%%%%%%%%%%%%%%%%%%%%%%%%%%%%%%%%%%%%%%%%%%%%%%%%%%
\subsubsection{Two level subsection examples}

This is a two-level subsection for exemplifying purposes.

%%%%%%%%%%%%%%%%%%%%%%%%%%%%%%%%%%%%%%%%%%%%%%%%%%%%%%%%%%%%%%%%%%%%%%%%%%%%%%%%%%%%%%%%%%%%%%%%%%%%%%%%%%%%%%%%%%%
\subsection{Page and margin sizes}

Page size is A4 (210 $\times$ 297~mm), and text is to be typeset in two columns, spaced 1.10~cm apart. Headers are different for even and odd pages (as is in this document). Left and right margins should be 1.90~cm, bottom margin is 2.00~cm and top margin is 2.10~cm too. Seek to use all the available area. Exceptions can be admitted, e.g. when it is required  to start a new section for instance. These can be allocated in beginning of the next page.

%%%%%%%%%%%%%%%%%%%%%%%%%%%%%%%%%%%%%%%%%%%%%%%%%%%%%%%%%%%%%%%%%%%%%%%%%%%%%%%%%%%%%%%%%%%%%%%%%%%%%%%%%%%%%%%%%%%
\subsection{Characters and Text}

The manuscript should use Times New Roman font, as provided by the template. The article's title is to be placed on the first page, flush-left using  18-pt \textbf{bold} font. Only its first letter is to be capitalized (except for proper names). Spacing after the title is 22-pt. The section titles should use  12-pt \textbf{bold} font, and should be completely capitalized, as presented in this template. Subsection titles use 12-pt \textbf{bold} font, and  only their first letter is to be capitalized (unless proper names are included). The running text must use simple spacing, 12-pt font, justified (aligned with both margins of the columns) and no indentation is to be used for the first line of every paragraph. Avoid the use of level three subsections, use a list system instead.

Make use of standard and scientific language in the text\footnote{Footnotes can help in clarifying minor details.}. Foreign words must be written in italic. Initials, acronyms, abbreviations and/or other compositions that escape from common knowledge should be presented to the reader, e.g., HRTF (\textit{Head-Related Transfer Function}) --- are always written using non-italic font, including in equations. Carry out a grammatical and technical revision before submission.

%%%%%%%%%%%%%%%%%%%%%%%%%%%%%%%%%%%%%%%%%%%%%%%%%%%%%%%%%%%%%%%%%%%%%%%%%%%%%%%%%%%%%%%%%%%%%%%%%%%%%%%%%%%%%%%%%%%
\subsection{Spacing between lines and paragraphs}

Simple spacing should be employed between lines, as adopted in this instructive document. Vertical separation between paragraphs is provided by the template. For manual adjustment Ms~Word users should chose the justified paragraph option (with 12-pt spacing).

%%%%%%%%%%%%%%%%%%%%%%%%%%%%%%%%%%%%%%%%%%%%%%%%%%%%%%%%%%%%%%%%%%%%%%%%%%%%%%%%%%%%%%%%%%%%%%%%%%%%%%%%%%%%%%%%%%%

\begin{figure*}[!ht] %% Example of a two column figure, 
	\centering
	\includegraphics[width=0.74\linewidth]{figs/Measurement-Scheme-Fonseca-2013.pdf}%
	\caption{\textit{Beamforming} measurement with cylindrical arrangement (adapted from Fonseca \cite{Fonseca-2013}).\\ Two-columns figure example.}%
	\label{fig:beamforming}%
\end{figure*}

%%%%%%%%%%%%%%%%%%%%%%%%%%%%%%%%%%%%%%%%%%%%%%%%%%%%%%%%%%%%%%%%%%%%%%%%%%%%%%%%%%%%%%%%%%%%%%%%%%%%%%%%%%%%%%%%%%%
\subsection{Equations, variables and units}

Equations should be inserted in the running text, with proper vertical separations, similar to the example of Equation ~\eqref{fig:beamforming}. Equations  should be centralized and  enumerated consecutively, being this numeration inserted flush right and between parentheses (see example). Remember that equations  are textual elements, therefore must be properly punctuated and the following text generally does not initiate with an upper case letter. It is recommended to introduce the nomenclature or definition of a variable right after the variable is presented in the equation.

When an already presented equation is to be cited in the text, one should do as follows: Equation~\eqref{eq:densidade} --- with only the first letter in upper case and with the respective number in parentheses.

The circle's area (in m$^2$) is given by
\begin{equation}
	A = \pi \, r^2\;,
\label{eq:area-circ}
\end{equation}
being $r$ the radius in meters (m). 
%
Remember that variables (like the $r$ in this example) are written in \textit{italic} (both in  equations, text, tables or figures). When in the running text, no parentheses should be used around the variable, because the variable's italic font makes it distinctive from the remainder of the text. 

However, \textbf{units, functions and mathematical operators} are to be written using non-italic font. For instance,"\ldots  32.0~N/m$^2$ was the applied pressure", or even
%
\begin{equation}
	\int_a^b p(\phi)\, \dt p\,,
\label{eq:int}
\end{equation}
was the calculated integral (notice that the differential operator ``d'' is using non-italic font), for each angle $\phi$ in degrees. As mathematical operators, one could mention the sine, $\sin(\theta)$, or the logarithmic operator $\log(y)$, for example.

Subscript or superscript text will only be in italic if corresponding to any pertinent variable. If it is a ``complementary name'' instead, the text shall be written upright, e.g., $P\txu{total}$ corresponds to the total pressure in Pa, or $S\txup{tri}$ corresponds to the triangle area in cm$^2$. However, regarding a variable, for example, $i$ one must write: the summation was calculated considering $P_i$ up to the \textit{i}-th final pressure corresponding to 256. Remember that the imaginary number i is a number, not a variable, and thus should not use italic font, not even in equations.

Text, initials or units used in equations should also not use italic font, e.g.,
\begin{equation}
	\text{density} = \frac{\tx{mass}}{\;\;\tx{volume}\;\;}\,,
\label{eq:densidade}
\end{equation}
being the kilogram  per cubic meter (kg/m$^3$) the unit of density in the SI system (International System of Units).

Units from the International System of Units (SI) should be adopted. When writing in Portuguese or Spanish, \textbf{use the comma decimal separator} in numbers, whether in text, tables, figures and/or graphics. In addition, make sure to use the same precision when comparing numbers, e.g.: 3.0 is different from 3.00 in terms of precision. However, it has same precision as 6.0. For texts written in English, it is up to the author whether to use dot or comma as the decimal separator (as long as the notations are not mixed). By writing a number and its unit\footnote{Units always use non-italic font, e.g., 30~N/m$^2$.}, always maintain the number along with the corresponding unit, without a line break between them (in Ms~Word, use Ctrl~+~Shift~+~Space, in \LaTeX, insert a tilde ($\sim$) between number and unit). For instance, a distance of 3~m separate the entrance from the exit, or 4.512~cm is the measured distance.

%%%%%%%%%%%%%%%%%%%%%%%%%%%%%%%%%%%%%%%%%%%%%%%%%%%%%%%%%%%%%%%%%%%%%%%%%%%%%%%%%%%%%%%%%%%%%%%%%%%%%%%%%%%%%%%%%%%
\subsection{Figures, tables and codes}

Figures, tables and codes shall be inserted along the text, by preference following the citing paragraphs that should include a  reference to the figures, tables, and codes respectively. Citation should be made before their actual presentation, for the reader's orientation. Interpretation of figures, tables and codes must be possible without reading the text itself. Figures and tables must be separated vertically from the text by a \textbf{single blank line} (12-pt). The \LaTeX{} template provides this separation automatically.

\begin{table*}[!b]
  \centering \ratb{1.3} 
  \caption{CPA 1 e CAUQ-B porous layers microgeometric and macrogeometric properties \cite{Mareze-2017}.\\ Two-column table example.}
	\fontsize{11}{12}\selectfont 
    \begin{tabular}{C{2.8cm} | C{1.5cm} | C{1.5cm} | C{1.5cm} | C{1.5cm} | C{1.5cm} | C{1.0cm}| C{1.0cm}}
    \toprule
		\SetRowColor{LightOrange}
    \textbf{ Samples / Parameter } & $L\txu{p}$ \qquad [$\upmu$\! m] & $L\txu{a}$ \qquad [$\upmu$\! m] & $D\txu{p}$ \qquad [$\upmu$\! m] & $D\txu{a}$ \qquad [$\upmu$\! m] & $\sigma$ [Ns/m\txup{4}] & {$\phi$\quad [--]} & $\alpha_{\infty}$ [--]\\
	  \midrule
		CPA 1 -  3\% &	1359,81 & 1492,51 & 2344,05 & 1425,67 &	5131 &	0,218 &	1,63\\
		\rowcolor[gray]{.95} CAUQ-B - 4,5\%	& 1598,29 &	701,24 & 2126,46 & 895,34 &	54989 &	0,070 &	2,89\\
    \bottomrule
    \end{tabular}
    \label{tab.exemplo}%
\end{table*}%

Figures, tables and codes must be horizontally centralized and numbered sequentially (see the examples in Figure~\ref{fig:beamforming}, \ref{subfig.exemplo} e \ref{fig:C80}; Table~\ref{tab.exemplo}; and  Code~\ref{code.matlalatex}). They may be inserted in one or two columns depending on their content. In case of two columns, it is recommended positioning them in the top or bottom of the page. Try to use figures and graphs that present fully comprehensive content. 

The figure's number and label, followed by the title should appear right below and centralized using 10-pt font. When content produced by other authors is used, even if adapted, indicate the source right after the descriptive title, as seen in the example given in  Figure~\ref{fig:beamforming}.

Numerations and titles of tables and codes must be placed above and centralized (see Table~\ref{tab.exemplo}). The table reference source (when necessary) must be presented according to the original publication. Table~\ref{tab.exemplo} is presented as an example for the style to be adopted. For the table's content a smaller font (smaller than 12-pt) can be used. Moreover, it is strongly recommended the use the automatic cross reference both in \LaTeX{} as in Ms~Word. Remember that all objects, like figures and tables, must be mentioned in text.

\begin{figure}[H]
  \centering \vspace{-1.5mm}
	\subfloat[Figure A]{\label{fig.figA}\includegraphics[height=23mm,page=2]{FIA-logo.pdf}}
	\qquad
  \subfloat[Figure B]{\label{fig.figB}\includegraphics[height=23mm,page=2]{FIA-logo.pdf}}	
  \caption{Side by side figures example.}
  \label{subfig.exemplo}
\end{figure}

\begin{figure}[ht!]
	\centering \vspace{-6mm}
        \includegraphics[width=0.98\linewidth,page=1]{figs/Combfilter-Brandao-2017.pdf}
        \caption{$C_{80}$ for distinct rooms. The figures can be inserted side by side (extracted from Brandão \cite{Brandao-2017}).}
	\label{fig:C80}%
\end{figure}


It is recommended that graphs, figures and any graph objects are inserted in \texttt{.jpg} and/or \texttt{.png} format with good quality (or even in vector form in \texttt{.pdf} for \LaTeX users). Make sure that graphic elements and figures are legible.

The distribution of this \LaTeX\xspace template includes the \texttt{Codes2Latex.sty} package\footnote{The package is still in development  and no detailed documentation is available. Hence, for further details, examine the \ttc{sty} file.}, which allows generic code documentation for codes from languages such as Matlab, Fortran, Python, LabView, and \LaTeX{} itself  in a organized form (see Code~\ref{code.matlalatex})

\begin{matlabcode}[Making Matlab Write Latex.]{code.matlalatex}
  syms x
  f = taylor(log(1+x));
  latex(f)
\end{matlabcode}

All elements (figures and graphs, for example) can be colored or in gray scale. Avoid the use of text elements from other authors without proper citation (and/or authorization). It is essential that text in figures is using the same language as the article. Indirect citations like the ones used in Google images, for example, will not be accepted, just as it is recommended to avoid the use of volatile knowledge bases.

All figures, tables and codes must be cross-referenced, for instance: Figure~\ref{fig:beamforming} and Table~\ref{tab.exemplo}. Note that the first letter is capitalized, because both the numbered figure as well as the numbered table is an object with a proper name . Also, the figure's or table's number should not be separated from the word Figure or Table to the next line. In Ms~Word, use Ctrl~+~Shift~+~Space, and in \LaTeX, insert a tilde ($\sim$) between the word Figure and the command \verb=\ref= or between the word Table and the command \verb=\ref=.

For sub-figures, use Figure~\subref*{fig.figA}, as an example.

%%%%%%%%%%%%%%%%%%%%%%%%%%%%%%%%%%%%%%%%%%%%%%%%%%%%%%%%%%%%%%%%%%%%%%%%%%%%%%%%%%%%%%%%%%%%%%%%%%%%%%%%%%%%%%%%%%%
\section{Article types}

Manuscripts should be \textbf{original submissions} (that is, not yet published) of scientific research and applied engineering, architecture, audio, physics, mathematics, speech and hearing science and related fields and sub-fields. Thus, the following document types will be considered:

\clearpage
\begin{itemize}[noitemsep,topsep=-1ex] \itemsep=7pt
	\item \textbf{Technical and applied papers}: present original material based on known and/or developing techniques. Applied methods that are in accordance with regulations and/or present pertinent results must be presented. It is essential that they are of interest to researchers and professionals in the area. 
	
	\item \textbf{Scientific papers}: contain original material (ideas, models, experiments, etc.) not published elsewhere, which substantially contribute to the scientific development. A relationship between the content and the already published \textit{state of the art} must be established.
	
    \item \textbf{Review papers}: discuss the \textit{state of the art} of the intended topic. This type of submission must aim for completeness, covering much of the already developed ideas, models, experiments, etc., even if they are in agreement with the author's opinion. It is important that the subject is of interest of the scientific community.
\end{itemize}

The thematic areas of the event include:

\begin{itemize}[noitemsep,topsep=0ex] \itemsep=3pt
    \item General Acoustics;
    \item Environmental Acoustics;
    \item Speech and Hearing Acoustics;
    \item Room Acoustics;
    \item Building Acoustics;
    \item Musical Acoustics;
    \item Underwater Acoustics;
    \item Vehicle Acoustics;
    \item Virtual Acoustics;
    \item Aeroacoustics;
    \item Audio and Electroacoustics;
    \item Bioacoustics;
    \item Noise Control;
    \item Teaching in Acoustics;
    \item Acoustics and Vibration measurements;
    \item Legislation and Standardization in Acoustics;
    \item Acoustic materials;
    \item Numerical methods applied to Acoustics and Vibrations;
    \item Soundscapes;
    \item Signal Processing;
    \item Psychoacoustics;
    \item Noise and Vibration in the work environment;
    \item Ultrasound;
    \item Vibroacoustics and Vibration; and
    \item INAD and IYS2020.
\end{itemize}

%%%%%%%%%%%%%%%%%%%%%%%%%%%%%%%%%%%%%%%%%%%%%%%%%%%%%%%%%%%%%%%%%%%%%%%%%%%%%%%%%%%%%%%%%%%%%%%%%%%%%%%%%%%%%%%%%%%
\section{Arrangement of the sections in the article}

The article structure should at least contemplate the following items:

\begin{itemize}[noitemsep,topsep=0ex] \itemsep=3pt
    \item Introduction: introduction of the subject, definition of objectives, clarification of relevance.
    \item Fundamentals: especially in scientific articles, the main theoretical foundation required for proper understanding of the reminder of the article must be presented and referenced.
    \item Development: how the work was realized, including theory, materials and methodological details.
    \item Results and discussions: partial or conclusive, according to the type of work
    \item Conclusion or final considerations: based on the discussion and objectives, arguments or considerations that conclude the study/application mus be presented.
    \item Acknowledgments: optional, if pertinent.
    \item References: list of references that have been cited in the text.
\end{itemize}

There is no strict necessity to use the names proposed herein for the sections. The arrangement of the sections can be different, depending on the article's type. Other post-textual elements like appendices are optional, as long as the total number of pages of the article, including post-textual elements, do not exceed 12 pages.

%%%%%%%%%%%%%%%%%%%%%%%%%%%%%%%%%%%%%%%%%%%%%%%%%%%%%%%%%%%%%%%%%%%%%%%%%%%%%%%%%%%%%%%%%%%%%%%%%%%%%%%%%%%%%%%%%%%
\subsection{Citations and references}

A separate section named \textbf{References}  must be inserted at the end of the document.

Both in the running text as well as in the reference list, all references should be \textbf{enumerated according to the  order they appear in the text}, using brackets \cite{Gomes-2015}. All references listed must be cited in the text. No uncited references should be added to the list of references. The references given in the template \cite{Mareze-2017,Fonseca-2013,Brandao-2017,Gomes-2015,Oppenheim-2010,Muller-2001,Mareze-2019,Borges-2018,Ristow-2016} are only illustrative.

All entries in the list of references must be formatted in 10-pt Tomes New Roman font, simple spaced and 6-pt paragraph spacing. This \LaTeX\xspace template uses the \texttt{natbib} package for the arrangement and formatting of the references. Moreover, the use of bibliography database managers like \href{http://www.jabref.org/}{JabRef}, \href{http://www.mendeley.com}{Mendeley} and \href{https://www.zotero.org/}{Zotero} is recommended. Specially for Word users, Mendeley has a plugin that formats and inserts the references in the \texttt{.docx} document.

Depending on the context, the name of the author may or may not be written when citing in the running text, according to the following examples:

\begin{itemize}[noitemsep,topsep=0ex] \itemsep=4pt
	\item 	``... Mareze et al.~\cite{Mareze-2019} worked in porous materials absorption...'' or
	\item ``... for the study of room acoustics~\cite{Brandao-2017}, it is recommended the reading of a textbook...'' or
	\item ``... applying the Fourier transform to the input signals \cite{Oppenheim-2010}. '' or even
	\item ``... Fonseca (2013) demonstrated the diffraction calculation for cylindrical surfaces~\cite{Fonseca-2013}.''
\end{itemize}

All authors appearing in the reference must be cited in text. 
For references with up to three authors, for example,  Müller e Massarani \cite{Muller-2001}, all authors must be cited (when evoked). In the case of more than three authors, for instance, Gomes \etal \cite{Gomes-2015}, only the first author's last name must be cited, followed the "\etal" expression. Still, by citing more than one reference, make use of just one bracket. Some examples are given as follows:

\begin{itemize}[noitemsep,topsep=0ex] \itemsep=8pt
	\item 	``Works in Vibration and Acoustics subjects \cite{Mareze-2017,Fonseca-2013,Brandao-2017}.''
	\item ``Works in Acoustics subjects \cite{Mareze-2017,Oppenheim-2010,Muller-2001,Mareze-2019}.''
	\item ``Works with statistical analysis \cite{Mareze-2017, Brandao-2017, Borges-2018}.''
	\item \textbf{Avoid this style:} ``Works with statistical analysis \cite{Mareze-2017}, \cite{Brandao-2017}, \cite{Ristow-2016}.''
\end{itemize}

Compacted and ordered references like \cite{Mareze-2017,Oppenheim-2010,Muller-2001,Mareze-2019} are recommended.

In the reference section, whenever possible, include ISBN, ISSN, DOI\footnote{For LaTeX users just provide the information in the  ``doi'' field of the  \texttt{.bib} file.} (with link) and/or link to the online address where the cited document is available.

%%%%%%%%%%%%%%%%%%%%%%%%%%%%%%%%%%%%%%%%%%%%%%%%%%%%%%%%%%%%%%%%%%%%%%%%%%%%%%%%%%%%%%%%%%%%%%%%%%%%%%%%%%%%%%%%%%%
\section{Submission and evaluation}

After the submitted abstract has been approved, the authors will be invited to elaborate the complete work. Details about registration and full manuscript submission can be found on the website \url{www.fia2022.com.br}, or can be obtained with the organizing committee.

It is the author's responsibility to submit the articles in their final form, as the organizing committee will not proceed with further adjustments. For this reason, authors are requested to verify the article's formatting with attention, specially graphs and figures, regarding their legitimacy and digital (and print) quality. \textbf{The articles should be sent in PDF format (with maximum file size of 10~Mb)}.

The PDF metadata for \LaTeX\xspace users are automatically generated. Ms~Word users must check during .docx to .pdf conversion.

Studies involving people (or living beings, in general), like in subjective acoustics or physiology, for instance, must inform the ethic committee approval term, if pertinent.

%%%%%%%%%%%%%%%%%%%%%%%%%%%%%%%%%%%%%%%%%%%%%%%%%%%%%%%%%%%%%%%%%%%%%%%%%%%%%%%%%%%%%%%%%%%%%%%%%%%%%%%%%%%%%%%%%%%
\section{Templates for Word and \LaTeX}

The \LaTeX\xspace template (\texttt{.tex}) was written in UTF8 encoding, thus being compatible with Windows, Mac, Linux and overleaf. It can be freely used for the article elaboration. We recommend \LaTeX{} users to access the template at \href{https://www.overleaf.com/read/hgryywpgmxdx}{Overleaf}\footnote{\url{https://www.overleaf.com/read/hgryywpgmxdx}.} and to create a copy to work on. On downloading the template and working offline with a local \TeX{} distribution, additional packages or fonts might be required and must be downloaded ans installed.

\clearpage
The author of the original template in Portuguese and models is professor William D'Andrea Fonseca, from Acoustical Engineering (EAC) of Federal University of Santa Maria (UFSM).
The Ms~Word version was created by Felipe Ramos de Mello (EAC/UFSM).
Translation to English has been carried out by Thiago Morphy and professor Stephan Paul (UFSC). 

All templates are available on the \href{http://www.fia2022.com.br}{event website}, \href{https://www.overleaf.com/read/hgryywpgmxdx}{Overleaf} (\href{https://www.overleaf.com/read/rnfjxkknksnd}{PT-BR}, \href{https://www.overleaf.com/read/rszcxtwshzfr}{SP} and \href{https://www.overleaf.com/read/hgryywpgmxdx}{EN}),  and \href{https://github.com/willdfonseca/fia2020}{GitHub}\footnote{\url{https://github.com/willdfonseca/fia2020}.}.

%%%%%%%%%%%%%%%%%%%%%%%%%%%%%%%%%%%%%%%%%%%%%%%%%%%%%%%%%%%%%%%%%%%%%%%%%%%%%%%%%%%%%%%%%%%%%%%%%%%%%%%%%%%%%%%%%%%
\section{Acknowledgments}

If pertinent, make acknowledgments. In case of work with financial support, use this section to elucidate details.

In the case of this document, we would like to thank everyone for their cooperation with the event.

%%%%%%%%%%%%%%%%%%%%%%%%%%%%%%%%%%%%%%%%%%%%%%%%%%%%%%%%%%%%%%%%%%%%%%%%%%%%%%%%%%%%%%%%%%%%%%%%%%%%%%%%%%%%%%%%%%%%
%%%%%%%%%%%%%%%%%%%%%%%%%%%%%%%%%%%%%%%%%%%%%%%%%%%%%%%%%%%%%%%%%%%%%%%%%%%%%%%%%%%%%%%%%%%%%%%%%%%%%%%%%%%%%%%%%%%%
%%% References
\renewcommand{\refname}{References} 
\bibliographystyle{unsrtnat} 
% \bibliographystyle{unsrtnat-br}  % Style similar to Brazilian standards
{\fontrefs \bibliography{bibliografia}}
%%%%%%%%%%%%%%%%%%%%%%%%%%%%%%%%%%%%%%%%%%%%%%%%%%%%%%%%%%%%%%%%%%%%%%%%%%%%%%%%%%%%%%%%%%%%%%%%%%%%%%%%%%%%%%%%%%%
\appendix
\section{Appendix example}

This is an appendix example. Additional information can be given here.

The \LaTeX\xspace template has some additional commands that make writing easier, like, for instance, \F\xspace to symbolize the Fourier Transform. For a better understanding of the commands, consult the \texttt{FIA2020.sty} file.

\end{document}
%%%%%%%%%%%%%%%%%%%%%%%%%%%%%%%%%%%%%%%%%%%%%%%%%%%%%%%%%%%%%%%%%%%%%%%%%%%%%%%%%%%%%%%%%%%%%%%%%%%%%%%%%%%%%%%%%%%
% EOF