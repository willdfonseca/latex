%%%%%%%%%%%%%%%%%%%%%%%%%%%%%%%%%%%%%%%%%%%%%%%%%%%%%%%%%%%%%%%%%%%%%%%%%%%%%%%%%%%%%%%%%%%%%%%%%%%%%%%%%%%%%%%%%%%%%%%%%%%%%%
%%% Template para de LaTeX para o evento XXX Encontro da Sociedade Brasileira de Acústica (https://www.even3.com.br/sobracnatal2023)
%%% Baseado no modelo da Revista Acústica e Vibrações da Sobrac (https://revista.acustica.org.br)
%%%	Desenvolvido por por Prof. William D'Andrea Fonseca, Dr. Eng. - Engenharia Acústica UFSM
%%% will.fonseca@eac.ufsm.br
%%% Release 31/03/2023
%%%%%%%%%%%%%%%%%%%%%%%%%%%%%%%%%%%%%%%%%%%%%%%%%%%%%%%%%%%%%%%%%%%%%%%%%%%%%%%%%%%%%%%%%%%%%%%%%%%%%%%%%%%%%%%%%%%%%%%%%%%%%%
\documentclass[12pt, a4paper, twoside, onecolumn]{article}
%%%%%%%%%%%%%%%%%%%%%%%%%%%%%%%%%%%%%%%%%%%%%%%%%%%%%%%%%%%%%%%%%%%%%%%%%%%%%%%%%%%%%%%%%%%%%%%%%%%%%%%%%%%%%%%%%%%%
\usepackage{Sobrac2023} %%% Template basics
%%%%%%%%%%%%%%%%%%%%%%%%%%%%%%%%%%%%%%%%%%%%%%%%%%%%%%%%%%%%%%%%%%%%%%%%%%%%%%%%%%%%%%%%%%%%%%%%%%%%%%%%%%%%%%%%%%%%
%%% Select language options
% Para artículos en español, cambie los nombres de los encabezados.
% For paper only in English uncomment the line below
% \SuppressResumo 
%%%%%%%%%%%%%%%%%%%%%%%%%%%%%%%%%%%%%%%%%%%%%%%%%%%%%%%%%%%%%%%%%%%%%%%%%%%%%%%%%%%%%%%%%%%%%%%%%%%%%%%%%%%%%%%%%%%%
%%% Dados do artigo

% \Authors{Fonseca,~W.~D'A.; Sobrenome,~N.} % Para metadata do PDF
\Authors{Fonseca,~W.~D'A.} % Para metadata do PDF

% Use a estrutura Sobrenome, N."
% \AuthorsAffiliations{Fonseca,~W.~D'A.$^1$; Sobrenome,~N.$^2$} % Use numeros para afiliações
\AuthorsAffiliations{Fonseca,~W.~D'A.$^1$} % Use numeros para afiliações

%% Use apenas UMA linha para autores com a mesma afiliação, declarando apenas diferentes emails
%% Quando houver autores de afiliações diferentes, use uma nova linha
% \Affiliations{ $^1$\,Engenharia Acústica, Universidade Federal de Santa Maria, Santa Maria, RS, Brasil, will.fonseca@eac.ufsm.br\\[2pt]  
% $^2$\,Laboratório de Vibrações, Instituição, Cidade, Estado, País, nome@dominio.br}

\Affiliations{ $^1$\,Engenharia Acústica, Universidade Federal de Santa Maria, Santa Maria, RS, Brasil, will.fonseca@eac.ufsm.br\\[2pt]}

\TitleComplete{Instruções e modelo de artigo para o XXX Encontro da\\ Sociedade Brasileira de Acústica (Sobrac)} % Título completo
\TitleShort{Instruções e modelo de artigo para o XXX Encontro da Sobrac}    % Título curto (se for necessário) para o cabeçalho da página
\TitleEnglish{Instructions and paper template for the XXX Meeting of the Brazilian Acoustics Society (Sobrac)}     % Title in English

\PalavrasChave{artigo técnico, Sobrac, acústica, vibrações} % Palavras-chave
\Keywords{technical paper, Sobrac, acoustics, vibration}    % Keywords

\Resumo{Este documento contém instruções para a escrita de artigos para o  o XXX Encontro da
Sociedade Brasileira de Acústica (Sobrac). 
%
Esse campo é destinado ao resumo do artigo que deve ter entre 180 e 300 palavras.\linebreak O resumo, palavras-chave, \textit{title}, \textit{abstract} e \textit{keywords} devem ser colocados na primeira página do artigo, buscando não se estender para a segunda página.  O resumo deve fazer uma apresentação concisa do artigo técnico científico, contendo, uma introdução, o objetivo, uma síntese da metodologia, o principal resultado e a principal conclusão (preferencialmente nessa ordem). Não é necessário separar em itens ou seções dentro do resumo. Assim, o leitor pode conhecer a essência do conteúdo do artigo. Lembre-se que o resumo é como o \textit{trailer} de um filme, as pessoas ficarão interessadas em ler completamente o artigo se o resumo lhes interessar. O resumo não deve conter informações novas não contidas no artigo; abreviações indefinidas; discussão prévia de outra literatura; referências e citações e excesso de detalhes acerca dos métodos empregados. Ele também não é o parágrafo de introdução do documento, isso deve ser colocado no início do texto. Utilize apenas informações úteis e relevantes, faça um exercício de empatia com o possível leitor interessado. Para se obter um resumo coeso, elegante e de acordo com o artigo, escreva uma prévia, realize a escrita completa do documento e, ao final, revise-o observando se o conteúdo dele reflete de forma consistente o teor do documento. Seguindo o resumo, o autor deve listar até cinco palavras-chave (evite colocar as mesmas palavras que formam o título do artigo). Após essa etapa, há ainda o título, resumo e palavras-chave em inglês.}

\Abstract{
This document contains instructions for writing papers for the XXX~Meeting of the Brazilian Society of Acoustics (Sobrac).
%
This field is intended for the abstract of the article, which must contain between 180 and 300 words. The elements \textit{resumo}, \textit{palavras-chave}, title, abstract, and keywords should constitute the first page (i.e., avoid extending them to the second page). The abstract should make a concise presentation of the scientific-technical article, containing an introduction, the objective, a synthesis of the methodology, the main result, and the final conclusion (preferably in that order). Separate items or sections are not required within the abstract. Thus, the reader may acknowledge the essence of the article content. Remember that the abstract is like a movie trailer, people will consider reading the complete article if the abstract is interesting. The abstract should not contain new information not contained within the article; undefined abbreviations; previous discussion of another literature; references and citations; or excessive detail about the methods employed. It is also not the introductory paragraph of the work; this should be placed at the beginning of the text. Use only relevant and useful information, exercising empathy with prospective readers. For a cohesive and elegant abstract that represents the article, write a preview, complete the paper, and then review it by looking at whether its content consistently reflects the content of the document. Following the abstract, the author should list up to five keywords (avoid using the same words contained in the article’s title). After this step, there are also title, abstract, and keywords in English.}

\Metadata % Includes the metada into the PDF
%%%%%%%%%%%%%%%%%%%%%%%%%%%%%%%%%%%%%%%%%%%%%%%%%%%%%%%%%%%%%%%%%%%%%%%%%%%%%%%%%%%%%%%%%%%%%%%%%%%%%%%%%%%%%%%%%%%%
%%%%%%%%%%%%%%%%%%%%%%%%%%%%%%%%%%%%%%%%%%%%%%%%%%%%%%%%%%%%%%%%%%%%%%%%%%%%%%%%%%%%%%%%%%%%%%%%%%%%%%%%%%%%%%%%%%%%
\begin{document} \setcounter{page}{1} %%%%%%%%%%%%%%%%%%%%%%%%%%%%%%%%%%%%%%%%%%%%%%%%%%%%%%%%%%%%%%%%%%%%%%%%%%%%%%%%%%%%%%%%%%%%%%%%%%%%%%%%%%%%%%%%%%%%
%%% Template de LaTeX para o Congresso Acústica 2020, que integra o XI Congreso Ibérico da Acústica, 
%%% 51º Congresso Espanhol de Acústica e TecniAcústica 2020
%%% Release 06/08/2020
%%%	Desenvolvido por Prof. William D'Andrea Fonseca da Engenharia Acústica, UFSM, Brasil
%%% will.fonseca@eac.ufsm.br
%%%%%%%%%%%%%%%%%%%%%%%%%%%%%%%%%%%%%%%%%%%%%%%%%%%%%%%%%%%%%%%%%%%%%%%%%%%%%%%%%%%%%%%%%%%%%%%%%%%%%%%%%%%%%%%%%%%%
%% Estilo do artigo
\pagestyle{plain}
%%%%%%%%%%%%%%%%%%%%%%%%%%%%%%%%%%%%%%%%%%%%%%%%%%%%%%%%%%%%%%%%%%%%%%%%%%%%%%%%%%%%%%%%%%%%%%%%%%%%%%%%%%%%%%%%%%%%
%%% Primeira página
\thispagestyle{firststyle}
%%%%%%%%%%%%%%%%%%%%%%%%%%%%%%%%%%%%%%%%%%%%%%%%%%%%%%%%%%%%%%%%%%%%%%%%%%%%%%%%%%%%%%%%%%%%%%%%%%%%%%%%%%%%%%%%%%%%
%%% Título
\begin{textblock}{210}(0,10)
\begin{figure}
\centering \includegraphics[height=2cm,page=1]{ac2020.pdf}
\end{figure}
\end{textblock}

\vspace{1cm}
\begin{center}
{\fontsize{14}{16}\selectfont\bfseries \MakeUppercase
%% Título
%%%%%%%%%%%%%%%%%%%%%%%%%%%%%%%%%%%%%%%%%%%%%%%%%%%%%%%%%%%%%%%%%%%%%%%%%%%%%%%%%%%%%%%%%%%%%%%%%%%%%%%%%%%%%%%%%%%%
\TituloCompletoArtigo
%%%%%%%%%%%%%%%%%%%%%%%%%%%%%%%%%%%%%%%%%%%%%%%%%%%%%%%%%%%%%%%%%%%%%%%%%%%%%%%%%%%%%%%%%%%%%%%%%%%%%%%%%%%%%%%%%%%%
\par}

%%%%%%%%%%%%%%%%%%%%%%%%%%%%%%%%%%%%%%%%%%%%%%%%%%%%%%%%%%%%%%%%%%%%%%%%%%%%%%%%%%%%%%%%%%%%%%%%%%%%%%%%%%%%%%%%%%%
%%%%%%%%%%%%%%%%%%%%%%%%%%%%%%%%%%%%%%%%%%%%%%%%%%%%%%%%%%%%%%%%%%%%%%%%%%%%%%%%%%%%%%%%%%%%%%%%%%%%%%%%%%%%%%%%%%%
\vspace{12pt}
{\fontsize{10}{12}\selectfont \bfseries 
%% Autores
%%%%%%%%%%%%%%%%%%%%%%%%%%%%%%%%%%%%%%%%%%%%%%%%%%%%%%%%%%%%%%%%%%%%%%%%%%%%%%%%%%%%%%%%%%%%%%%%%%%%%%%%%%%%%%%%%%%
\AutoresFiliacoesArtigo
%%%%%%%%%%%%%%%%%%%%%%%%%%%%%%%%%%%%%%%%%%%%%%%%%%%%%%%%%%%%%%%%%%%%%%%%%%%%%%%%%%%%%%%%%%%%%%%%%%%%%%%%%%%%%%%%%%%
\par}

\vspace{2mm}
{\fontsize{10}{11}\selectfont 
%% Filiações
%%%%%%%%%%%%%%%%%%%%%%%%%%%%%%%%%%%%%%%%%%%%%%%%%%%%%%%%%%%%%%%%%%%%%%%%%%%%%%%%%%%%%%%%%%%%%%%%%%%%%%%%%%%%%%%%%%%
\FiliacoesArtigo
%%%%%%%%%%%%%%%%%%%%%%%%%%%%%%%%%%%%%%%%%%%%%%%%%%%%%%%%%%%%%%%%%%%%%%%%%%%%%%%%%%%%%%%%%%%%%%%%%%%%%%%%%%%%%%%%%%%%
\par}
\end{center}
%%%%%%%%%%%%%%%%%%%%%%%%%%%%%%%%%%%%%%%%%%%%%%%%%%%%%%%%%%%%%%%%%%%%%%%%%%%%%%%%%%%%%%%%%%%%%%%%%%%%%%%%%%%%%%%%%%%%
%%%%%%%%%%%%%%%%%%%%%%%%%%%%%%%%%%%%%%%%%%%%%%%%%%%%%%%%%%%%%%%%%%%%%%%%%%%%%%%%%%%%%%%%%%%%%%%%%%%%%%%%%%%%%%%%%%%%
%%%%%%%%%%%%%%%%%%%%%%%%%%%%%%%%%%%%%%%%%%%%%%%%%%%%%%%%%%%%%
%%% Resumo e palavras-chave
%%%%%%%%%%%%%%%%%%%%%%%%%%%%%%%%%%%%%%%%%%%%%%%%%%%%%%%%%%%%%
		{\fontsize{11}{13}\selectfont 
		\textbf{Resumo}
		\vspace{-2mm}
		
		\ResumoArtigo
		\par}
		%%%%%%%%%%%%%%%%%%%%%%%%% Palavras-chave
		\vspace{0.1\baselineskip} \fontsize{11}{13}\selectfont 
		\textbf{Palavras-chave: }{\fontsize{11}{13}\selectfont 
		\PalavrasChaveArtigo.
		\par}
%%%%%%%%%%%%%%%%%%%%%%%%%%%%%%%%%%%%%%%%%%%%%%%%%%%%%%%%%%%%%%%%%%%%%%%%%%%%%%%%%%%%%%%%%%%%%%%%%%%%%%%%%%%%%%%%%%%%
%%%%%%%%%%%%%%%%%%%%%%%%%%%%%%%%%%%%%%%%%%%%%%%%%%%%%%%%%%%%%%%%%%%%%%%%%%%%%%%%%%%%%%%%%%%%%%%%%%%%%%%%%%%%%%%%%%%%
\vspace{0.5\baselineskip}
%%%%%%%%%%%%%%%%%%%%%%%%%%%%%%%%%%%%%%%%%%%%%%%%%%%%%%%%%%%%%
%%% Abstract and keywords
%%%%%%%%%%%%%%%%%%%%%%%%%%%%%%%%%%%%%%%%%%%%%%%%%%%%%%%%%%%%%
\begin{otherlanguage*}{english}
%%%%%%%%%%%%%%%%%%%%%%%%%%%%%%%%%%%%%%%%%%%%%%%%%%%%%%%%%%%%%%%%%%%%%%%%%%%%%%%%%%%%%%%%%%%%%%%%%%%%%%%%%%%%%%%%%%%%
%%%%%%%%%%%%%%%%%%%%%%%%%%%%%%%%%%%%%%%%%%%%%%%%%%%%%%%%%%%%%%%%%%%%%%%%%%%%%%%%%%%%%%%%%%%%%%%%%%%%%%%%%%%%%%%%%%%%
%%% Abstract
		{\fontsize{11}{13}\selectfont 
		\textbf{Abstract}
		\vspace{-2mm}
		
		\AbstractArtigo
		\par}
		%%%%%%%%%%%%%%%%%%%%%%%%% Keywords
		\vspace{0.1\baselineskip} \fontsize{11}{13}\selectfont 
		\textbf{Keywords: }{\fontsize{11}{13}\selectfont 
		\KeywordsArtigo.
		\par}
\end{otherlanguage*}		
%%%%%%%%%%%%%%%%%%%%%%%%%%%%%%%%%%%%%%%%%%%%%%%%%%%%%%%%%%%%%%%%%%%%%%%%%%%%%%%%%%%%%%%%%%%%%%%%%%%%%%%%%%%%%%%%%%%%
%%%%%%%%%%%%%%%%%%%%%%%%%%%%%%%%%%%%%%%%%%%%%%%%%%%%%%%%%%%%%%%%%%%%%%%%%%%%%%%%%%%%%%%%%%%%%%%%%%%%%%%%%%%%%%%%%%%%
%%%%%%%%%%%%%%%%%%%%%%%%% PACs:
		\vspace{0.1\baselineskip}
		{\fontsize{11}{13}\selectfont \bfseries
		\textbf{PACS no. }
		\PACSArtigo.
		\par}
%%%%%%%%%%%%%%%%%%%%%%%%%%%%%%%
\vspace{1\baselineskip}
% EOF %%%%%%%%%%%%%%%%%%%%%%%%%


  % Início do documento
%%%%%%%%%%%%%%%%%%%%%%%%%%%%%%%%%%%%%%%%%%%%%%%%%%%%%%%%%%%%%%%%%%%%%%%%%%%%%%%%%%%%%%%%%%%%%%%%%%%%%%%%%%%%%%%%%%%%
%%%%%%%%%%%%%%%%%%%%%%%%%%%%%%%%%%%%%%%%%%%%%%%%%%%%%%%%%%%%%%%%%%%%%%%%%%%%%%%%%%%%%%%%%%%%%%%%%%%%%%%%%%%%%%%%%%%%
%%% ARTIGO
%%%%%%%%%%%%%%%%%%%%%%%%%%%%%%%%%%%%%%%%%%%%%%%%%%%%%%%%%%%%%%%%%%%%%%%%%%%%%%%%%%%%%%%%%%%%%%%%%%%%%%%%%%%%%%%%%%%%
%%%%%%%%%%%%%%%%%%%%%%%%%%%%%%%%%%%%%%%%%%%%%%%%%%%%%%%%%%%%%%%%%%%%%%%%%%%%%%%%%%%%%%%%%%%%%%%%%%%%%%%%%%%%%%%%%%%%

%%%%%%%%%%%%%%%%%%%%%%%%%%%%%%%%%%%%%%%%%%%%%%%%%%%%%%%%%%%%%%%%%%%%%%%%%%%%%%%%%%%%%%%%%%%%%%%%%%%%%%%%%%%%%%%%%%%%
%%%%%%%%%%%%%%%%%%%%%%%%%%%%%%%%%%%%%%%%%%%%%%%%%%%%%%%%%%%%%%%%%%%%%%%%%%%%%%%%%%%%%%%%%%%%%%%%%%%%%%%%%%%%%%%%%%%%
%%% ARTIGO
%%%%%%%%%%%%%%%%%%%%%%%%%%%%%%%%%%%%%%%%%%%%%%%%%%%%%%%%%%%%%%%%%%%%%%%%%%%%%%%%%%%%%%%%%%%%%%%%%%%%%%%%%%%%%%%%%%%%
%%%%%%%%%%%%%%%%%%%%%%%%%%%%%%%%%%%%%%%%%%%%%%%%%%%%%%%%%%%%%%%%%%%%%%%%%%%%%%%%%%%%%%%%%%%%%%%%%%%%%%%%%%%%%%%%%%%%
\clearpage % Recomenda-se deixar apenas os dados de resumo da primeira página, porém, isso não é obrigatório.

\section{Introdução}

Este texto de instruções modelo foi elaborado para que os autores possam apresentar os artigos de forma padronizada. 
Ele foi adaptado do modelo da \href{https://revista.acustica.org.br}{Revista Acústica e Vibrações}, sendo de uso para o XXX Encontro da Sociedade Brasileira de Acústica (Sobrac).
%
Isso proporcionará uma uniformidade da formatação para os artigos completos do evento.
Neste modelo são apresentadas as principais diretrizes para a elaboração do artigo completo no que diz respeito à apresentação de conteúdo, gráfica, estrutura, diagramação e ao procedimento para a submissão dos artigos. 
Este documento já conta com a formatação de estilos personalizados para a elaboração do artigo. O autor pode, portanto, utilizar este arquivo como modelo para essa finalidade. Serão disponibilizados modelos (\textit{templates}) em Microsoft Word (\texttt{.docx}) e \LaTeX\xspace (\texttt{.tex}). Esta versão também está disponível no \href{https://www.overleaf.com/read/xnhkrtjwprcn}{Overleaf} e no \href{https://github.com/willdfonseca/latex}{GitHub} --- sendo ainda compatível com Windows, Mac e Linux. 
Os autores são responsáveis pelo conteúdo, elaboração e envio dos artigos de acordo com o presente modelo.

O texto completo deverá estar em espaçamento simples entre linhas, tipografia Times New Roman tamanho 12~pt e parágrafo com espaçamento de 0~pt antes e 12~pt depois. É prática comum a escrita de artigos científicos no impessoal, logo,  isso é recomendado. Além disso, serão aceitos em língua culta\footnote{Faça uso de corretores ortográficos e/ou de gramática, tanto Ms Word quanto o Overleaf possuem, é indicado ainda o uso de outras ferramentas como o \href{https://languagetool.org/pt-BR}{Language Tool}.} portuguesa, inglesa\footnote{Artigos em língua estrangeira escritos por não-nativos devem, preferencialmente, receber revisão profissional.} e espanhola\footnotemark[2]. 


%%%%%%%%%%%%%%%%%%%%%%%%%%%%%%%%%%%%%%%%%%%%%%%%%%%%%%%%%%%%%%%%%%%%%%%%%%%%%%%%%%%%%%%%%%%%%%%%%%%%%%%%%%%%%%%%%%%
%%%%%%%%%%%%%%%%%%%%%%%%%%%%%%%%%%%%%%%%%%%%%%%%%%%%%%%%%%%%%%%%%%%%%%%%%%%%%%%%%%%%%%%%%%%%%%%%%%%%%%%%%%%%%%%%%%%
\section{Orientações básicas}

Nesta seção há um resumo de como o artigo deve ser construído. Para mais detalhes, consulte as seções subsequentes.

\vspace{-8pt}
\begin{enumerate} \itemsep=2pt
    \item Os modelos em LaTeX e Word fornecidos já contêm todas as configurações descritas neste documento. Além disso, este manuscrito fornece simultaneamente instruções para as duas plataformas de diagramação de texto.
	\item A primeira página deve conter (para língua portuguesa) título, autores, filiações, resumo, palavras-chave, \textit{title}, \textit{abstract} e \textit{keywords}.
	Submissões em espanhol devem ter itens similares, porém em língua espanhola. Submissões em inglês podem conter apenas \textit{title}, \textit{abstract} e \textit{keywords}.
	\item O texto deve ser escrito em língua culta vigente.
	\item O número máximo de páginas é 12, contando da página que contém o título, até o final das referências (incluindo apêndices, se houver).
	\item O tamanho do papel é A4, com margens: superior de 2,0~cm, inferior de 2,0~cm, esquerda de 1,8~cm e direita de 1,8~cm (o espaçamento entre colunas é de 1,0~cm).
	\item O texto deve ser escrito com tipografia Times New Roman com tamanho 12~pt (conforme este modelo).
	\item O artigo pode conter figuras, tabelas, quadros, códigos e equações. No texto, caso sejam necessários, links podem ser colocados. Animações também são aceitas, desde que estejam diagramadas como figuras.
	\item Entende-se que um artigo técnico tenha uma estrutura lógica, descritiva e conteúdo passível de reprodução, findando nas referências do trabalho.
\end{enumerate}


%%%%%%%%%%%%%%%%%%%%%%%%%%%%%%%%%%%%%%%%%%%%%%%%%%%%%%%%%%%%%%%%%%%%%%%%%%%%%%%%%%%%%%%%%%%%%%%%%%%%%%%%%%%%%%%%%%%
%%%%%%%%%%%%%%%%%%%%%%%%%%%%%%%%%%%%%%%%%%%%%%%%%%%%%%%%%%%%%%%%%%%%%%%%%%%%%%%%%%%%%%%%%%%%%%%%%%%%%%%%%%%%%%%%%%%
\section{Documento e apresentação}

Sempre coloque texto em seções e subseções, não as deixe órfãs (abrindo uma seção e passando direto para a subseção).

%%%%%%%%%%%%%%%%%%%%%%%%%%%%%%%%%%%%%%%%%%%%%%%%%%%%%%%%%%%%%%%%%%%%%%%%%%%%%%%%%%%%%%%%%%%%%%%%%%%%%%%%%%%%%%%%%%%
\subsection{Primeira página}

A primeira página deve conter os seguintes itens colocados pelos autores: título, autores, filiações, resumo, palavras-chave, \textit{title}, \textit{abstract} e \textit{keywords}. 
%
Caso o título completo seja muito extenso, pede-se uma versão curta para que seja incluída no cabeçalho das páginas do artigo.


O resumo do artigo poderá ter entre 180 e 300 palavras (em tipografia de 11~pt). O resumo, palavras-chave, \textit{title}, \textit{abstract} e \textit{keywords} constituem a primeira página do artigo, é recomendado não se estender para outra página. 
Ele deve fazer uma apresentação concisa do artigo técnico científico, contendo uma introdução, o objetivo, uma síntese da metodologia, o principal resultado e a principal conclusão (preferencialmente nessa ordem). Não é necessário separar em itens ou seções dentro do resumo. Assim, o leitor pode conhecer a essência do trabalho. Lembre-se que o resumo é como o \textit{trailer} de um filme, as pessoas ficarão interessadas em ler completamente o artigo se o resumo lhes interessar. O resumo não deve conter informações novas não contidas no artigo; abreviações indefinidas; discussão prévia de outra literatura; referências e citações e excesso de detalhes acerca dos métodos empregados. Ele também não é o parágrafo de introdução do documento, isso deve ser colocado no início do texto. Utilize apenas informações úteis e relevantes, faça um exercício de empatia com o possível leitor interessado. Para se obter um resumo coeso, elegante e de acordo com o artigo, escreva uma prévia, realize a escrita completa do documento e, ao final, revise-o observando se o conteúdo dele reflete de forma consistente o teor do documento. 

Seguindo o resumo, o autor deve listar até cinco palavras-chave (evite colocar as mesmas palavras que formam o título do artigo). O artigo deve começar propriamente na segunda página do artigo (Introdução \etc).

Na filiação dos autores use números como marcas e caso existam autores de uma mesma instituição, utilize apenas um endereço e os diferencie nos emails. Quando existirem emails de um mesmo domínio, busque reduzir usando chaves \{\}. Utilize no máximo duas linhas para a filiação de cada autor de instituições diferentes. Veja a seguir alguns  exemplos:
%
\begin{flushleft}
\vspace{-0.25\baselineskip}
\begin{itemize}[topsep=-1ex,align=left,leftmargin=0.2cm] \itemsep=4pt

	\item Fonseca,~W.~D'A.$^1$; Sobrenome,~N.$^2$\\[6pt]	
	$^{1,2}$\,Engenharia Acústica, Universidade Federal de Santa Maria, Santa Maria, RS, Brasil, 
	 will.fonseca@eac.ufsm.br, nome@dominio.br.
	
	\item Fonseca,~W.~D'A.$^1$; Mareze,~P.~H.$^2$\\[6pt]	
	$^{1-2}$\,Engenharia Acústica, Universidade Federal de Santa Maria, Santa Maria, RS, Brasil,\\
	\{will.fonseca, paulo.mareze\}@eac.ufsm.br.
	
	\item Fonseca,~W.~D'A.$^1$; Sobrenome,~N.$^2$, Mareze,~P.~H.$^3$\\[6pt]	
	$^{1,3,2}$\,Engenharia Acústica, Universidade Federal de Santa Maria, Santa Maria, RS, Brasil,\\
	\{will.fonseca, paulo.mareze\}@eac.ufsm.br, nome@dominio.br.

	\item Fonseca,~W.~D'A.$^1$; Sobrenome,~N.$^2$\\[6pt]	
	$^{1}$\,Engenharia Acústica, Universidade Federal de Santa Maria, Santa Maria, RS, Brasil,
	will.fonseca@eac.ufsm.br.\\[4pt]		
	$^2$\,Laboratório de Vibrações, Instituição, Cidade, Estado, País, nome@dominio.br.	
\end{itemize}
\vspace{-0.4\baselineskip}
\end{flushleft}

	
%%%%%%%%%%%%%%%%%%%%%%%%%%%%%%%%%%%%%%%%%%%%%%%%%%%%%%%%%%%%%%%%%%%%%%%%%%%%%%%%%%%%%%%%%%%%%%%%%%%%%%%%%%%%%%%%%%%
\subsection{Número de páginas}

O trabalho completo deve conter de 6 a 12 páginas, contando da  página que contém o título e o final da lista de referências. São admitidos apêndices, depois das referências, desde que estes não ultrapassem 12 páginas no total. 

Como forma de otimizar ao máximo o conteúdo de cada página, as figuras, tabelas, quadros e códigos devem ser apresentados ao longo do corpo do texto (em uma ou duas colunas, dependendo de seu conteúdo).

%%%%%%%%%%%%%%%%%%%%%%%%%%%%%%%%%%%%%%%%%%%%%%%%%%%%%%%%%%%%%%%%%%%%%%%%%%%%%%%%%%%%%%%%%%%%%%%%%%%%%%%%%%%%%%%%%%%
\subsubsection{Exemplo de subseção de dois níveis}

Esta é uma subseção de dois níveis para efeito de exemplificação.

%%%%%%%%%%%%%%%%%%%%%%%%%%%%%%%%%%%%%%%%%%%%%%%%%%%%%%%%%%%%%%%%%%%%%%%%%%%%%%%%%%%%%%%%%%%%%%%%%%%%%%%%%%%%%%%%%%%
\subsection{Tamanho da folha e margens}

O texto deve ser configurado em folha do tamanho A4 (210 $\times$ 297~mm), em uma coluna, com numeração distinta de páginas pares e ímpares (como está neste documento). As margens esquerda e direita deverão ter 1,8~cm, a inferior 2,0~cm e a superior 2,0~cm. Procure utilizar toda a área disponível. Exceções podem ser admitidas, por exemplo, quando for necessário começar uma nova seção, título, subtítulo ou legenda: esses poderão ser alocados no início da página seguinte.

%%%%%%%%%%%%%%%%%%%%%%%%%%%%%%%%%%%%%%%%%%%%%%%%%%%%%%%%%%%%%%%%%%%%%%%%%%%%%%%%%%%%%%%%%%%%%%%%%%%%%%%%%%%%%%%%%%%
\subsection{Caracteres e texto}

Os textos deverão ser escritos em tipografia Times New Roman. O título do artigo deverá estar na primeira página, centralizado, \textbf{em negrito}, com apenas a primeira letra em maiúscula (exceto nomes próprios), corpo 18~pt e parágrafo com espaço de 22~pt depois. Os títulos das seções deverão ser em negrito, corpo 12~pt, com apenas a primeira letra em maiúsculo (a não ser que existam nomes próprios), conforme apresentado neste modelo. As subseções devem ser também em negrito, corpo 12~pt, para ambos os casos, utilize tipografia Times New Roman. O texto do documento deve ter espaçamento simples, corpo 12~pt, justificado e sem recuo na primeira linha. Evite o uso de subseções com mais de três níveis e, para isso, busque usar um sistema de listas. 

% No Latex isso já está configurado automaticamente.

Utilize linguagem culta e científica em seu texto\footnote{Notas de rodapé podem ajudar a aclarar detalhes e comentários.}. Palavras estrangeiras deverão ser grafadas em itálico (por exemplo, como em \textit{proceedings}). Siglas, acrônimos, abreviaturas e/ou outras construções que fogem ao conhecimento comum devem ser apresentadas ao leitor, por exemplo, HRTF (\textit{Head-Related Transfer Function}) --- são sempre grafados "em pé", inclusive em equações.
Faça revisões gramaticais e de cunho técnico antes da submissão.

%%%%%%%%%%%%%%%%%%%%%%%%%%%%%%%%%%%%%%%%%%%%%%%%%%%%%%%%%%%%%%%%%%%%%%%%%%%%%%%%%%%%%%%%%%%%%%%%%%%%%%%%%%%%%%%%%%%
\subsection{Espaçamento entre linhas e parágrafos}

Deve-se empregar espaçamento simples entre linhas, como já adotado neste arquivo de instruções.
Na formatação dos parágrafos escolher a opção parágrafo justificado (com espaçamento de 12~pt).

% No Latex isso já está configurado automaticamente.

%%%%%%%%%%%%%%%%%%%%%%%%%%%%%%%%%%%%%%%%%%%%%%%%%%%%%%%%%%%%%%%%%%%%%%%%%%%%%%%%%%%%%%%%%%%%%%%%%%%%%%%%%%%%%%%%%%%
\subsection{Equações e unidades}

Serão adotadas as unidades do Sistema Internacional (SI). Ao escrever seu trabalho em português ou espanhol, nos números, \textbf{use o separador decimal vírgula} (conforme a língua portuguesa e espanhola vigente), seja no texto, tabelas, figuras e/ou gráficos, além de buscar sempre o uso de uma mesma precisão ao comparar números, por exemplo: 3,0 é diferente de 3,00, porém tem a mesma precisão de 6,0. 
No caso do trabalho ser escrito em inglês, fica a critério do autor usar ponto ou vírgula como separador decimal (desde que não misture as notações).
Ao escrever um número com sua unidade\footnote{Unidades são sempre grafadas ``em pé'', ou seja, não em itálico, por exemplo, 30~N/m$^2$.}, mantenha sempre o número junto à correspondente unidade, sem que exista quebra de linha entre eles (no Ms Word utilize Ctrl + Shift + Espaço [ou Alt + 0160], no \LaTeX\xspace coloque um til ($\sim$) entre o número e a unidade). Por exemplo, 3~m de distância separa a entrada e a saída e 4.512,28~cm é a distância medida.

As equações deverão estar encaixadas entre o texto (no Word use uma ``tabela'' simples) conforme o exemplo da Equação~\eqref{eq:area-circ}. Deverão ainda estar centralizadas e numeradas sequencialmente, com a numeração colocada no lado direito e entre parênteses (vide exemplo). Lembre-se que elas são elementos textuais, logo, devem ser pontuadas e o texto conseguinte normalmente não se inicia com letra maiúscula. Recomenda-se colocar a nomenclatura imediatamente após a variável apresentada.

A área do círculo (em m$^2$) é dada por 
\begin{equation}
	A = \pi \, r^2\;,
\label{eq:area-circ}
\end{equation}
%
em que $r$ é o raio em metros (m) --- nota: escrever ``onde'' depois de equações é identificado como erro. Lembre-se que variáveis (como o $r$ nesse exemplo) são grafadas em \textit{itálico} (seja na equação ou no texto). Porém, \textbf{unidades, funções e operadores matemáticos são escritos ``em pé''}, sem a aplicação do itálico. Por exemplo, 32,0~N/m$^2$ foi a pressão aplicada, ou ainda
%
\begin{equation}
	\int_a^b p(\phi)\, \dt p\,
\label{eq:int}
\end{equation}
%
foi a integral calculada (observe que o operador diferencial ``$\tx{d}$'' está em pé), para cada ângulo $\phi$ em graus. Como funções, pode-se citar o seno, $\sen(\theta)$, ou ainda $\log(y)$, por exemplo. 
%

Texto subscrito e sobrescrito somente será em itálico se for correspondente a alguma variável pertinente. Caso seja um ``nome complementar'', o texto deve ser colocado em pé, por exemplo, $P\txu{total}$ corresponde à pressão total em Pa, ou ainda $S\txup{tri}$ corresponde à área do triângulo em cm$^2$. Porém, em se tratando de uma variável, por exemplo, $i$ deve-se escrever: o somatório foi calculado considerando $P_i$ até a $i$-ésima pressão final correspondente a 256.

Caso texto, siglas ou unidades sejam utilizados em equações, sua representação deve ser em pé, por exemplo:
%
\begin{equation}
	\text{densidade} = \frac{\tx{massa}}{\;\;\tx{volume}\;\;}\,,
\label{eq:densidade}
\end{equation}
%
sendo que no SI (Sistema Internacional de Unidades) a unidade de densidade é o quilograma por metro cúbico (kg/m$^3$).
%
No texto, quando for necessário citar uma equação já apresentada, deve-se fazê-lo da seguinte forma: Equação~\eqref{eq:densidade} --- com apenas a primeira letra em maiúsculo e com o número correspondente entre parênteses.

%%%%%%%%%%%%%%%%%%%%%%%%%%%%%%%%%%%%%%%%%%%%%%%%%%%%%%%%%%%%%%%%%%%%%%%%%%%%%%%%%%%%%%%%%%%%%%%%%%%%%%%%%%%%%%%%%%%
\subsection{Figuras, tabelas, quadros e códigos}

As figuras e tabelas devem ser inseridas durante o texto, preferencialmente em seguida aos parágrafos a que se referem. Uma menção
às figuras, tabelas, quadros e códigos no texto corrido, antes da sua apresentação, é necessária para a orientação do leitor. As figuras, tabelas e quadros devem conter todos os elementos de formatação e de conteúdo para que sejam interpretados corretamente, sem necessidade de se recorrer ao texto corrido para uma busca de informações adicionais. Deve-se separar do texto as tabelas e figuras com \textbf{1 linha} em branco antes e depois (12~pt). 

\enlargethispage{1em}

% No Latex isso já está configurado automaticamente.

As figuras, tabelas e quadros deverão ser centralizados e numerados sequencialmente (vide exemplo nas Figuras~\ref{fig:beamforming} e \ref{subfig.exemplo}; Tabela~\ref{tab.exemplo}; Quadro~\ref{quad.exemplo} e Código~\ref{code.matlalatex}). Busque utilizar figuras e gráficos em que seu conteúdo possa ser completamente compreendido. 

\begin{figure}[!ht] %% Exemplo de figura em duas colunas
	\centering
	\includegraphics[width=0.72\linewidth]{figs/Measurement-Scheme-Fonseca-2013.pdf}%
	\caption{Medição de \textit{beamforming} com arranjo cilíndrico (adaptado de Fonseca \cite{Fonseca-2013}) --- exemplo de figura.}
	\label{fig:beamforming}%
\end{figure}

\begin{table}[!b]
  \centering \ratb{1.3} 
  \caption{Propriedades microgeométricas e macroscópicas das camadas porosas CPA 1 e CAUQ-B\\ (retirado de Mareze \etal \cite{Mareze-2017}) --- exemplo de tabela.}
	\fontsize{11}{12}\selectfont 
    \begin{tabular}{C{2.8cm} | C{1.5cm} | C{1.5cm} | C{1.5cm} | C{1.5cm} | C{1.5cm} | C{1.0cm}| C{1.0cm}}
    \toprule
		\SetRowColor{LightOrange}
    \textbf{ Amostra / Parâmetro } & $L\txu{p}$ \qquad [$\upmu$\! m] & $L\txu{a}$ \qquad [$\upmu$\! m] & $D\txu{p}$ \qquad [$\upmu$\! m] & $D\txu{a}$ \qquad [$\upmu$\! m] & $\sigma$ [Ns/m\txup{4}] & {$\phi$\quad [--]} & $\alpha_{\infty}$ [--]\\
	  \midrule
		CPA 1 -  3\% &	1359,81 & 1492,51 & 2344,05 & 1425,67 &	5131 &	0,218 &	1,63\\
		\rowcolor[gray]{.95} CAUQ-B - 4,5\%	& 1598,29 &	701,24 & 2126,46 & 895,34 &	54989 &	0,070 &	2,89\\
    \bottomrule
    \end{tabular}
    \label{tab.exemplo}%
\end{table}%

O rótulo e número das figuras, seguido da legenda, deve aparecer logo abaixo e centralizado (10~pt). Caso utilize figuras de outros autores (ou fontes), mesmo que adaptadas, indique a fonte logo após a legenda descritiva, vide exemplo da Figura~\ref{fig:beamforming}.

O rótulo, número e legenda das tabelas (quadros e códigos também) devem aparecer centralizados na parte superior (vide Tabela~\ref{tab.exemplo}). A fonte das tabelas deve ser apresentada de acordo com a publicação original (quando necessário). A Tabela~\ref{tab.exemplo} apresenta um exemplo do estilo a ser utilizado (o conteúdo da tabela poderá conter tipografia menor que a do texto quando necessário). Ademais, recomenda-se fortemente o sistema de referências cruzadas automatizado. \textbf{Lembre-se que todos os objetos, como figuras e tabelas, devem ser citados no texto.}

\begin{figure}[!ht]
  %\ContinuedFloat %% para continuar a partir da figura anterior
  \centering
	\subfloat[Legenda da Figura~\protect\subref*{fig.figA}.]{\label{fig.figA}
            \makebox[.45\columnwidth]{\includegraphics[height=35mm,page=48]{example-image-duck}} } % makebox ajuda a organizar figs lado a lado
	\quad
  \subfloat[Legenda da Figura~\protect\subref*{fig.figB}.]{\label{fig.figB}
            \makebox[.45\columnwidth]{\includegraphics[height=35mm,page=47]{example-image-duck}} }
  \caption{Exemplo de figuras lado a lado.}
  \label{subfig.exemplo}
\end{figure}


\begin{quadro}[ht!]
%\vspace{-2mm}
  \centering \ratb{1.3} \setlength\aboverulesep{0pt} \setlength\belowrulesep{0pt}
  \caption{Este é um exemplo de um quadro.}
    \fontsize{11}{12}\selectfont 
    \begin{tabular}{| C{2.8cm} | C{1.8cm} | C{1.8cm} |}
    \hline
	\SetRowColor{LightBlue}
    \textbf{ Experimento / Tipo } & \textbf{Exp. 1} & \textbf{Exp. 2}\\
	\midrule
		Tipo 1& Verde & Amarela\\
		\rowcolor[gray]{.95} Tipo 2 & Azul & Branco\\
	\hline
    \end{tabular}
    \label{quad.exemplo}%
    \vspace{2mm}
\end{quadro}%


%% Modelo de figuras lado a lado usando minipage
%\begin{figure*}[b]
    %\centering
    %\begin{minipage}[t]{.48\textwidth}
        %\centering
        %\includegraphics[width=1\linewidth,page=2]{FIA-logo.pdf}
        %\caption{Figura do lado esquerdo.}
        %\label{fig:ladoE}
    %\end{minipage}%
		%\quad
    %\begin{minipage}[t]{0.48\textwidth}
        %\centering
        %\includegraphics[width=1\linewidth,page=2]{FIA-logo.pdf}
        %\caption{Figura do lado direito.}
        %\label{fig:ladoD}
    %\end{minipage}
%\end{figure*}

Recomenda-se que gráficos, figuras, fotos e qualquer arquivo gráfico, estejam inseridos no texto em formato .jpg e/ou .png com boa qualidade (ou ainda em formato vetorial em .pdf para usuários do \LaTeX\xspace). Atente para que os elementos de gráficos e figuras sejam legíveis (sobretudo se a informação for pertinente).


A distribuição deste \textit{template} de \LaTeX\xspace inclui o pacote \ttc{Codes2Latex.sty}\footnote{O pacote está ainda em desenvolvimento (sem documentação detalhada), logo, para mais detalhes, consulte o arquivo \ttc{sty}.}, que habilita possibilidades para documentação de códigos genéricos e nas linguagens Matlab, Fortran, Python, LabView e Latex de forma organizada (observe o Código~\ref{code.matlalatex}).

\clearpage
\begin{matlabcode}[Exemplo de um excerto de código (fazendo o Matlab escrever Latex).]{code.matlalatex}
  syms x
  f = taylor(log(1+x));
  latex(f)
\end{matlabcode}

Todos os elementos (figuras e gráficos, por exemplo) podem ser coloridos ou em tons de cinza. Evite a utilização de elementos textuais de outros autores sem a devida citação (e/ou autorização). É essencial que as figuras que apresentarem texto estejam na mesma língua do artigo. Não serão aceitas citações indiretas como \textit{Google Imagens}, por exemplo, assim como recomenda-se evitar o uso de bases de conhecimento voláteis.

\enlargethispage{2mm}

As referências cruzadas devem ser feitas para todos os elementos, por exemplo: Figura~\ref{fig:beamforming} e Tabela~\ref{tab.exemplo} (apenas a primeira letra maiúscula). Caso exista uma subfigura, use Figura~\subref*{fig.figA}, por exemplo.


%%%%%%%%%%%%%%%%%%%%%%%%%%%%%%%%%%%%%%%%%%%%%%%%%%%%%%%%%%%%%%%%%%%%%%%%%%%%%%%%%%%%%%%%%%%%%%%%%%%%%%%%%%%%%%%%%%%
\section{Tipos de artigo}

O evento aceitará \textbf{submissões originais} (isto é, ainda não publicadas) de pesquisas científicas e aplicações de engenharia, arquitetura, áudio, física, matemática, fonoaudiologia e áreas (e subáreas) afins. Assim, serão considerados os seguintes tipos de documento:
%
\begin{itemize}[noitemsep,topsep=0ex] \itemsep=7pt
	\item \textbf{Artigos técnicos e aplicados} (\textit{Technical and applied papers}):
	apresentam material original a partir de aplicações de técnicas conhecidas e/ou em desenvolvimento. Deve apresentar métodos aplicados que estejam de acordo com normativas e/ou que apresentem resultados pertinentes. É essencial que sejam de interesse de pesquisadores e profissionais do tema proposto.
	
	\item \textbf{Artigos científicos} (\textit{Scientific papers}): 
	contém material original (ideias, modelos, experimentos \etc) não publicado, que contribui substancialmente para o avanço da ciência naquele tema. Ele deve estabelecer uma relação entre seu conteúdo e o \textit{estado da arte} já publicado. 

	\item \textbf{Artigos de revisão} (\textit{Review papers}):
	discutem o \textit{estado da arte} sobre o tema pretendido, aclarando desde aspectos básicos até os sofisticados. Esse tipo de submissão deve ser completo no que concerne à literatura, cobrindo em boa parte as ideias, modelos, experimentos \etc já desenvolvidos, mesmo que não estejam de acordo com a opinião do autor. É importante que o assunto seja de interesse da comunidade científica.	
\end{itemize}

\vspace{5pt}

As áreas temáticas do evento incluem:
%	
\begin{itemize}[noitemsep,topsep=-1ex] \itemsep=1.5pt
\item[\textbullet] Acústica geral; 
\item[\textbullet] Acústica ambiental
\item[\textbullet] Acústica da audição e da fala;
\item[\textbullet] Acústica de salas;
\item[\textbullet] Acústica de edificações;
\item[\textbullet] Acústica musical;
\item[\textbullet] Acústica submarina;
\item[\textbullet] Acústica veicular;
\item[\textbullet] Acústica virtual e técnica biauricular;
\item[\textbullet] Processamento de sinais;
\item[\textbullet] Técnicas de imageamento acústico;
\item[\textbullet] Aeroacústica;
\item[\textbullet] Áudio e eletroacústica;
\item[\textbullet] Bioacústica;
\item[\textbullet] Controle de ruído;
\item[\textbullet] Ensino em acústica;
\item[\textbullet] Medições/instrumentação em acústica e vibrações;
\item[\textbullet] Legislação e normalização em acústica;
\item[\textbullet] Materiais acústicos;
\item[\textbullet] Métodos numéricos em acústica e vibrações;
\item[\textbullet] Paisagens sonoras;
\item[\textbullet] Psicoacústica (acústica fisiológica);
\item[\textbullet] Acústica subjetiva;
\item[\textbullet] Ruído e vibrações em ambiente laboral;
\item[\textbullet] Ultrassom; 
\item[\textbullet] Vibrações e vibroacústica;
\item[\textbullet] Fonoaudiologia, audiologia e temas relacionados à saúde; e
\item[\textbullet] INAD e ações de extensão.
\end{itemize}
	

%%%%%%%%%%%%%%%%%%%%%%%%%%%%%%%%%%%%%%%%%%%%%%%%%%%%%%%%%%%%%%%%%%%%%%%%%%%%%%%%%%%%%%%%%%%%%%%%%%%%%%%%%%%%%%%%%%%
%%%%%%%%%%%%%%%%%%%%%%%%%%%%%%%%%%%%%%%%%%%%%%%%%%%%%%%%%%%%%%%%%%%%%%%%%%%%%%%%%%%%%%%%%%%%%%%%%%%%%%%%%%%%%%%%%%%
\section{Organização do trabalho}

A estrutura do artigo deverá contemplar pelo menos os seguintes itens:
%
\begin{itemize}[noitemsep,topsep=0ex] \itemsep=3pt
	\item Introdução: visão geral sobre o assunto com definição dos objetivos do trabalho, indicando a sua relevância.
	\item Fundamentos: sobretudo em artigos científicos, a fundamentação teórica principal necessária ao entendimento do texto deve ser apresentada e referenciada. 
	\item Desenvolvimento: como o trabalho foi realizado, incluindo detalhes de teoria, materiais e métodos empregados.
	\item Resultados e discussões: parciais ou conclusivos, conforme a modalidade do trabalho, fazendo referência a medições e cálculos estatísticos aplicados, se for o caso.
	\item Conclusões ou Considerações finais: basear-se nas discussões e objetivos, apresentando apontamentos e considerações que findam o estudo/aplicação.
	\item Agradecimentos: opcional, quando for pertinente. Nessa seção admite-se ainda declarações acerca de financiamento de pesquisa/projeto.
	\item Referências: apresentar bibliografia citada no texto.
\end{itemize}
%
Não é preciso necessariamente existir seções com estes nomes. A organização é também dependente do tipo do artigo.
Outros elementos pós-textuais como apêndices são opcionais, desde que eles (no total) não excedam o limite total de 12 páginas. 

%%%%%%%%%%%%%%%%%%%%%%%%%%%%%%%%%%%%%%%%%%%%%%%%%%%%%%%%%%%%%%%%%%%%%%%%%%%%%%%%%%%%%%%%%%%%%%%%%%%%%%%%%%%%%%%%%%%
\subsection{Citações e referências}

Para a confecção das referências deve-se utilizar a norma vigente. As referências devem ser \textbf{numeradas conforme ordem de aparição}, utilizando colchetes \cite{Gomes-2015}. Todas referências devem ser citadas durante o texto. As referências \cite{Mareze-2017,Fonseca-2013,Brandao-2017,Gomes-2015,Oppenheim-2010,Muller-2001,Mareze-2019,aev:piccini2020} deste modelo de artigo são apenas ilustrativas (para efeito de compreensão).

Ao final do documento a seção de referências deve ser colocada. As entradas nela contidas devem ter tipografia com tamanho 10~pt, espaçamento simples e espaçamento de parágrafo de 6~pt. Este \textit{template} de \LaTeX\xspace usa o pacote {\ttfamily abntex2cite} (com melhorias) para a organização das referências. Além disso, recomenda-se a utilização de gerenciadores de banco de dados de bibliografia como o \href{http://www.jabref.org/}{JabRef}, \href{http://www.mendeley.com}{Mendeley} e \href{https://www.zotero.org/}{Zotero}. Em especial para usuários do Word, o Mendeley tem um \textit{plugin} para formatar e inserir as referências no documento .docx.


Dependendo do contexto, o nome do autor pode ou não ser escrito, observe os exemplos a seguir: 
%
\begin{itemize}[noitemsep,topsep=0ex] \itemsep=4pt
	\item 	``... Mareze \etal \cite{Mareze-2019} trabalharam com absorção de materiais porosos...'' ou 
	
	\item ``... para o estudo de acústica de salas \cite{Brandao-2017} recomenda-se a leitura de um livro texto...'' ou
	\item ``... aplicando a Transformada de Fourier nos sinais de entrada \cite{Oppenheim-2010}. '' ou ainda
	\item ``... Fonseca (2013) demonstrou o cálculo de difração para superfícies cilíndricas~\cite{Fonseca-2013}.''
\end{itemize}
%
Todos os autores que constam nas referências devem estar citados no texto.

Em referências com até três autores, por exemplo, Müller e Massarani \cite{Muller-2001}, ambos devem ser citados (quando evocados). No caso de mais de três autores, por exemplo, Gomes \etal \cite{Gomes-2015} deve-se citar somente o último nome do primeiro autor seguido da expressão ``\etal''. Ainda, ao citar mais de uma referência, utilize apenas um colchete, veja alguns exemplos a seguir:
%
\begin{itemize}[noitemsep,topsep=0ex] \itemsep=8pt
	\item ``Trabalhos em temas de acústica e vibrações \cite{Mareze-2017,Fonseca-2013,Brandao-2017}.''
	\item ``Trabalhos em temas de acústica \cite{Mareze-2017,Oppenheim-2010,Muller-2001,sobrac2018:natal, Mareze-2019, jasa:2022eac}.''
	\item ``Trabalhos com análise estatística \cite{Mareze-2017, Brandao-2017, aev:piccini2020}.''
		\item \textbf{Não usar esse estilo:} ``Trabalhos com análise estatística \cite{Mareze-2017}, \cite{Brandao-2017}, \cite{jasa:2022eac} ou \cite{Mareze-2017}--\cite{jasa:2022eac}.''
\end{itemize}
%
Recomenda-se que as referências sejam ordenadas e compactadas (com meia-risca) como em \cite{Mareze-2017,Oppenheim-2010,Muller-2001,Mareze-2019}.

Na seção de referências, sempre que possível, inclua o ISBN, ISSN, DOI\footnote{Para usuários de Latex basta usar o campo ``doi'' de seu \texttt{.bib}.} (com link) e/ou link com a direção online em que o documento citado está disponível.

%%%%%%%%%%%%%%%%%%%%%%%%%%%%%%%%%%%%%%%%%%%%%%%%%%%%%%%%%%%%%%%%%%%%%%%%%%%%%%%%%%%%%%%%%%%%%%%%%%%%%%%%%%%%%%%%%%%
\section{Submissão e avaliação}

Os artigos completos deverão ser enviados pelo sistema próprio do Encontro, disponível no site\linebreak \url{https://www.even3.com.br/sobracnatal2023}, dentro dos prazos estabelecidos. Os autores serão comunicados e receberão o parecer dos avaliadores (em pares) do trabalho. Após atender as correções solicitadas, quando for o caso, o artigo deverá ser reenviado pelo mesmo sistema, seguindo as condições de reenvio. Detalhes acerca de registro autor participante podem ser consultados também no site, ou com a comissão organizadora.

Detalhes acerca de registro do autor participante podem ser consultados também no site, ou com a comissão organizadora.

É responsabilidade dos autores a preparação e envio dos artigos em seu formato final. Por esse motivo, pede-se que verifiquem com atenção a formatação de seus artigos, especialmente gráficos e fotos, quanto à legibilidade e qualidade digital (e para impressão). \textbf{Os artigos deverão ser enviados em formato PDF (com tamanho máximo de 10~Mb).} 

Os metadados do PDF para usuários de \LaTeX\xspace são feitos automaticamente, usuários de MS Word devem conferir no momento da conversão.

% Uso de PDF-a fica opcional.

Em pesquisas que envolvam pessoas (ou seres vivos, em geral), como em acústica subjetiva ou fisiológica, por exemplo, recomenda-se aclarar no artigo o termo de aprovação do Comitê de Ética, caso pertinente.



%%%%%%%%%%%%%%%%%%%%%%%%%%%%%%%%%%%%%%%%%%%%%%%%%%%%%%%%%%%%%%%%%%%%%%%%%%%%%%%%%%%%%%%%%%%%%%%%%%%%%%%%%%%%%%%%%%%
\subsection{Modelos para Word e \LaTeX}

O modelo de \LaTeX\xspace (\texttt{.tex}) foi escrito em codificação UTF8, assim é compatível com Windows, Mac, Linux e \href{https://www.overleaf.com/read/xnhkrtjwprcn}{Overleaf}\footnote{\url{https://www.overleaf.com/read/xnhkrtjwprcn}.}. Pode ser usado livremente para a elaboração dos artigos.

O modelo de \texttt{.docx} foi criado em Microsoft Word 2016 e, com isso, suas funcionalidades de espaçamento e configurações são garantidas para essa versão. Todas eles estão disponíveis com links no \href{https://www.even3.com.br/sobracnatal2023}{site do evento}.

O autor deste texto e dos modelos é o professor William D'Andrea Fonseca, da Engenharia Acústica (EAC) da Universidade Federal de Santa Maria (UFSM).

%%%%%%%%%%%%%%%%%%%%%%%%%%%%%%%%%%%%%%%%%%%%%%%%%%%%%%%%%%%%%%%%%%%%%%%%%%%%%%%%%%%%%%%%%%%%%%%%%%%%%%%%%%%%%%%%%%%
\section{Considerações finais}

Buscou-se, por meio desse \textit{artigo modelo}, elencar e aclarar as instruções para submissão de artigos para o\linebreak XXX Encontro da Sobrac. 
Este próprio documento pode ser usado como modelo apenas trocando o conteúdo.

%%%%%%%%%%%%%%%%%%%%%%%%%%%%%%%%%%%%%%%%%%%%%%%%%%%%%%%%%%%%%%%%%%%%%%%%%%%%%%%%%%%%%%%%%%%%%%%%%%%%%%%%%%%%%%%%%%%
\section{Agradecimentos}

Se for pertinente, faça agradecimentos.
%
Em caso de trabalhos com fomento, utilize esta seção para elucidar detalhes.

No caso deste documento, gostaríamos de agradecer à cooperação de todos para com o evento.
%%%%%%%%%%%%%%%%%%%%%%%%%%%%%%%%%%%%%%%%%%%%%%%%%%%%%%%%%%%%%%%%%%%%%%%%%%%%%%%%%%%%%%%%%%%%%%%%%%%%%%%%%%%%%%%%%%%
% EOF

%%%%%%%%%%%%%%%%%%%%%%%%%%%%%%%%%%%%%%%%%%%%%%%%%%%%%%%%%%%%%%%%%%%%%%%%%%%%%%%%%%%%%%%%%%%%%%%%%%%%%%%%%%%%%%%%%%%%
%%%%%%%%%%%%%%%%%%%%%%%%%%%%%%%%%%%%%%%%%%%%%%%%%%%%%%%%%%%%%%%%%%%%%%%%%%%%%%%%%%%%%%%%%%%%%%%%%%%%%%%%%%%%%%%%%%%%
%%% Referências
\renewcommand{\refname}{Referências} \addcontentsline{toc}{section}{\refname}% 
\bibliographystyle{abntex2-num} 
{\footnotesize \bibliography{bibliografia}}
%%%%%%%%%%%%%%%%%%%%%%%%%%%%%%%%%%%%%%%%%%%%%%%%%%%%%%%%%%%%%%%%%%%%%%%%%%%%%%%%%%%%%%%%%%%%%%%%%%%%%%%%%%%%%%%%%%%
%%%%%%%%%%%%%%%%%%%%%%%%%%%%%%%%%%%%%%%%%%%%%%%%%%%%%%%%%%%%%%%%%%%%%%%%%%%%%%%%%%%%%%%%%%%%%%%%%%%%%%%%%%%%%%%%%%%
\appendix
\section{Exemplo de apêndice}

Este é um exemplo de apêndice, geralmente se colocam informações adicionais ou derivações produzidas pelos autores.

Este modelo (\textit{template} de \LaTeX) tem alguns comandos adicionais que facilitam a escrita, como, por exemplo, \lcode{\\F} (\F\xspace) para simbolizar a Transformada de Fourier. Para conhecer melhor os comandos, consulte o arquivo \texttt{Sobrac2023.sty}.

%%%%%%%%%%%%%%%%%%%%%%%%%%%%%%%%%%%%%%%%%%%%%%%%%%%%%%%%%%%%%%%%%%%%%%%%%%%%%%%%%%%%%%%%%%%%%%%%%%%%%%%%%%%%%%%%%%%
%%%%%%%%%%%%%%%%%%%%%%%%%%%%%%%%%%%%%%%%%%%%%%%%%%%%%%%%%%%%%%%%%%%%%%%%%%%%%%%%%%%%%%%%%%%%%%%%%%%%%%%%%%%%%%%%%%%
% Exemplo de códigos em duas colunas
%\onecolumn
%\section{Escrevendo um código de apêndice em uma coluna}
%
% Caso existam códigos muito extensos, o apêndice pode ser convertido para uma coluna, como exemplificado nesta seção com o Código~\ref{code.fc}.
%
%\inputcodeLatin{Matlab}{Exemplo de inclusão de código (de Matlab) em duas colunas.}{code.fc}{FC.m}

%%%%%%%%%%%%%%%%%%%%%%%%%%%%%%%%%%%%%%%%%%%%%%%%%%%%%%%%%%%%%%%%%%%%%%%%%%%%%%%%%%%%%%%%%%%%%%%%%%%%%%%%%%%%%%%%%%%
% \clearpage  \listoftodos % Lista temporária que pode ajudar na escrita do artigo.
%%%%%%%%%%%%%%%%%%%%%%%%%%%%%%%%%%%%%%%%%%%%%%%%%%%%%%%%%%%%%%%%%%%%%%%%%%%%%%%%%%%%%%%%%%%%%%%%%%%%%%%%%%%%%%%%%%%
%%%%%%%%%%%%%%%%%%%%%%%%%%%%%%%%%%%%%%%%%%%%%%%%%%%%%%%%%%%%%%%%%%%%%%%%%%%%%%%%%%%%%%%%%%%%%%%%%%%%%%%%%%%%%%%%%%%
\end{document}
%%%%%%%%%%%%%%%%%%%%%%%%%%%%%%%%%%%%%%%%%%%%%%%%%%%%%%%%%%%%%%%%%%%%%%%%%%%%%%%%%%%%%%%%%%%%%%%%%%%%%%%%%%%%%%%%%%%
% EOF